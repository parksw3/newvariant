\documentclass[12pt]{article}
\usepackage[top=1in,left=1in, right = 1in, footskip=1in]{geometry}

\usepackage{graphicx}
\usepackage{xspace}
%\usepackage{adjustbox}

\newcommand{\comment}{\showcomment}
%% \newcommand{\comment}{\nocomment}

\newcommand{\showcomment}[3]{\textcolor{#1}{\textbf{[#2: }\textsl{#3}\textbf{]}}}
\newcommand{\nocomment}[3]{}

\newcommand{\jd}[1]{\comment{cyan}{JD}{#1}}
\newcommand{\swp}[1]{\comment{magenta}{SWP}{#1}}
\newcommand{\bmb}[1]{\comment{blue}{BMB}{#1}}
\newcommand{\djde}[1]{\comment{red}{DJDE}{#1}}

\newcommand{\eref}[1]{Eq.~\ref{eq:#1}}
\newcommand{\fref}[1]{Fig.~\ref{fig:#1}}
\newcommand{\Fref}[1]{Fig.~\ref{fig:#1}}
\newcommand{\sref}[1]{Sec.~\ref{#1}}
\newcommand{\frange}[2]{Fig.~\ref{fig:#1}--\ref{fig:#2}}
\newcommand{\tref}[1]{Table~\ref{tab:#1}}
\newcommand{\tlab}[1]{\label{tab:#1}}
\newcommand{\seminar}{SE\mbox{$^m$}I\mbox{$^n$}R}

\usepackage{amsthm}
\usepackage{amsmath}
\usepackage{amssymb}
\usepackage{amsfonts}

\usepackage{lineno}
\linenumbers

\usepackage[pdfencoding=auto, psdextra]{hyperref}

\usepackage{natbib}
\bibliographystyle{apalike}
\date{\today}

\usepackage{xspace}
\newcommand*{\ie}{i.e.\@\xspace}

\usepackage{color}

%% Consistent, changeable style for subscripts
\newcommand{\vvvar}{\mathrm{var}}
\newcommand{\wwwt}{\mathrm{wt}}

\newcommand{\rx}[1]{\ensuremath{{r}_{#1}}\xspace} 
\newcommand{\ry}[1]{\rx{\mathrm{#1}}} 
\newcommand{\rw}{\rx{\wwwt}}
\newcommand{\rv}{\rx{\vvvar}}

\newcommand{\Rx}[1]{\ensuremath{{\mathcal R}_{#1}}\xspace} 
\newcommand{\Ry}[1]{\Rx{\mathrm{#1}}}
\newcommand{\Ro}{\Rx{0}}
\newcommand{\RR}{\ensuremath{{\mathcal R}}\xspace}
\newcommand{\Rw}{\Rx{\wwwt}}
\newcommand{\Rv}{\Rx{\vvvar}}

\newcommand{\days}{\ensuremath{\, \textrm{days}}}
\newcommand{\pday}{\ensuremath{/\textrm{day}}}
\newcommand{\dd}[1]{\ensuremath{\, \mathrm{d}#1}}
\newcommand{\dtau}{\dd{\tau}}
\newcommand{\dx}{\dd{x}}
\newcommand{\dsigma}{\dd{\sigma}}

\newcommand{\ix}[1]{\ensuremath{{i}_{#1}}\xspace} 
\newcommand{\iy}[1]{\ix{\mathrm{#1}}}
\newcommand{\iw}{\ix{\wwwt}}
\newcommand{\iv}{\ix{\vvvar}}

\newcommand{\Gx}[1]{\ensuremath{{\bar G}_{#1}}\xspace} 
\newcommand{\Gy}[1]{\Gx{\mathrm{#1}}}
\newcommand{\Gw}{\Gx{\wwwt}}
\newcommand{\Gv}{\Gx{\vvvar}}

\newcommand{\tsub}[2]{#1_{{\textrm{\tiny #2}}}}
\newcommand{\tstart}{\ensuremath{\tsub{t}{start}}\xspace}
\newcommand{\tend}{\ensuremath{\tsub{t}{end}}\xspace}

\newcommand{\betaeff}{\ensuremath{\tsub{\beta}{eff}}\xspace}
\newcommand{\Keff}{\ensuremath{\tsub{K}{eff}}\xspace}

\newcommand{\pt}{p} %% primary time
\newcommand{\st}{s} %% secondary time

\newcommand{\psize}{{\mathcal P}} %% primary cohort size
\newcommand{\ssize}{{\mathcal S}} %% secondary cohort size

\newcommand{\gtime}{\sigma} %% generation interval
\newcommand{\gdist}{g} %% generation-interval distribution

\newcommand{\geff}{g_{\textrm{eff}}} %% generation-interval distribution

\newcommand{\total}{{\mathcal T}} %% total number of serial intervals

\newcommand{\PP}{{\mathcal P}}
\newcommand{\II}{{\mathcal I}}

\begin{document}

\begin{flushleft}{
	\Large
	\textbf\newline{
		Roles of generation-interval distributions in characterizing relative epidemic strength and speed of new SARS-CoV-2 variants and their control
	}
}
\end{flushleft}

\section*{Abstract}

Inferring the relative strength (i.e., the ratio of reproduction numbers, $\Rv/\Rw$) and relative speed (i.e., the difference between growth rate, $\rv-\rw$) of new SARS-CoV-2 variants compared to their wild types is critical to predicting and controlling the course of the current pandemic.
Multiple studies have estimated the relative strength of new variants from the observed relative speed, but they typically neglect the possibility that the new variants have different generation intervals (i.e., time between infection and transmission), which determines the relationship between relative strength and speed.
However, the B.1.1.7 strain may have a longer infectious period (and therefore, generation interval) than prior dominant lineages.
Here, we explore how differences in generation intervals between a new variant and the wild type affect the relationship between relative strength and speed.
We use simulations to show how neglecting their differences can lead to biases and illustrate how such biases can be assessed in practice.
Finally, we discuss implications for control: if new variants have longer generation intervals then speed-like interventions (e.g., contact tracing) become more effective, whereas strength-like interventions become less effective (e.g., social distancing).

\section{Introduction}

Estimating variant epidemic strength and speed remains one of the key questions in controlling the spread and understanding the threat of SARS-CoV-2 variants of concern (VoCs) \citep{switzerland2021variant, davies2021estimated, di2021impact, leung2021early, volz2021transmission,zhao2021}.
Epidemic strength is characterized by the reproduction number $\RR$, which determines the average number of new infections per infection as well as the strength of intervention required to eliminate the disease \citep{anderson1991infectious}.
It also provides information about the final size of an epidemic in a homogeneous population \citep{anderson1991infectious}.
Epidemic speed is characterized by the growth rate $r$, which describes how fast a disease spreads at the population-level.
Like epidemic strength, epidemic speed also determines conditions for disease elimination $r=0$ is a threshold equivalent to $\RR=1$ \citep{doi:10.1098/rspb.2020.1556}.
Strength and speed are linked by generation intervals---defined as time between infection and transmission---which characterizes the individual-level time scale of the epidemic \citep{roberts2007model,svensson2007note,wallinga2007generation}.

Analyses of new variants have typically characterized \emph{relative} strength (ratios between reproduction numbers $\Rv/\Rw$) and speed (differences between growth rates $\rv-\rw$) of the variants.
Many studies have tried to estimate the relative strength of the novel variant from the observed relative speed \citep{davies2021estimated, leung2021early, volz2021transmission,zhao2021}.
Some studies have instead assumed a value of the relative strength of the novel variant and tried to predict its relative speed to determine when a new variant will become the dominant strain. \citep{davies2021estimated,di2021impact}.
While both approaches are reasonable in principle, holding different quantities constant (i.e., strength or speed) can lead to different conclusions about the spread of the disease and its control \citep{doi:10.1098/rspb.2020.1556}:
estimation of speed from strength or strength from speed depends on assumptions about the generation-interval distribution \citep{roberts2007model,svensson2007note,wallinga2007generation}.
For example, when we assume a fixed value of epidemic strength, longer generation intervals lead to a slower epidemic (lower $r$), making the epidemic seem easier to control.
When we assume a fixed value of epidemic speed, longer generation intervals lead to stronger epidemic (higher $\RR$), making the epidemic seem more difficult to control.

In practice, most studies have assumed that the previous dominant strain (wild type) and variants have identical generation intervals, but recent evidence suggests that the B.1.1.7 variant may have a longer duration of infection: 13.3 days (90\% CI: 10.1--16.5) for the new variant and 8.2 days (90\% CI: 6.5--9.7) for the wild type \citep{kissler2021densely}.
Longer duration of infection implies that the mean generation interval of B.1.1.7 is likely to be longer than that of the wild type.
Other studies have considered the possibility that the faster growth rate of new variants may be driven, in part, by shorter generation intervals \citep{davies2021estimated,volz2021transmission}.
In general, if the generation-interval distribution of a new variant is different from the generation-interval distribution of the wild type, estimates of variant strength may be biased.
However, linking strength and speed is complicated given that generation intervals depend on many other biological and behavioral factors;
for example, self-isolation after symptom onset will lead to shorter generation intervals.

Here, we explore how different assumptions affect estimates of the relative strength $\RR$ and speed $r$ for a new variant (e.g., B.1.1.7).
We compare the relationship between relative strength (the ratio $\Rv/\Rw$) and relative speed (the difference $\rv-\rw$) under a wide range of assumptions about generation-interval distributions.
We find that neglecting differences in the generation-interval distributions can lead to biased estimates.
We also discuss how such biases might be assessed in practice and how information on differences in generation interval distributions might also influence priorities for controlling the spread of VoCs.

\section{Renewal equation framework}

We use the renewal equation framework to characterize the spread of two pathogen strains---in this case, the wild type SARS-CoV-2 virus and a focal variant of concern, such as B.1.1.7.
Neglecting the (relatively slow) rate of new mutations, the current incidence of infection $i_x(t)$ caused by strain $x$---wild type (``wt'') and the variant (``var'')---can be expressed in terms of their previous incidence $i_x(t-\tau)$ and the rate at which secondary cases are generated at time $t$ by individuals infected $\tau$ time units ago $K_x(t, \tau)$:
\begin{equation}
i_x(t) = \int_0^\infty i_x(t-\tau) K_x(t, \tau) \dtau.
\end{equation}
This framework provides a flexible way of modeling disease dynamics and generalizes compartmental models, such as the SEIR model \citep{heesterbeek1996concept, diekmann2000mathematical, roberts2004modelling, aldis2005integral, roberts2007model, champredon2018equivalence}.

The integral of the kernel $\RR_x(t) = \int K_x(t, \tau) \dtau$ is referred to as the instantaneous reproduction number \citep{fraser2007estimating}.
The instantaneous reproduction number is a particular kind of weighted average which describes the average infectiousness of previously infected individuals at time $t$---in particular, it is weighted by their relative infectiousness at time $t$, rather than by the actual number of infected individuals present.
The normalized kernel $g_x(t, \tau) = K_x(t, \tau)/\RR_x(t)$---which we refer to as the instantaneous generation-interval distribution---describes the relative contribution of previously infected individuals to current incidence $i_x(t)$ and provides information about the time scale of disease transmission.
Both the reproduction number and the generation-interval distribution depend on many factors, including intrinsic infectiousness of an infected individual, non-pharmaceutical interventions, awareness-driven behavior, and population-level susceptibility \citep{fraser2007estimating}.

Over a short period of time, we can assume that epidemiological conditions remain roughly constant: $\RR_x(t) \approx \RR_x$ and $g_x(t, \tau) \approx g_x(\tau)$.
In this case, the incidence of each strain will exhibit exponential growth (or decay) at rate $r_x$, satisfying the Euler-Lotka equation \citep{wallinga2007generation}:
\begin{equation}
\frac{1}{\RR_x} = \int_0^\infty \exp(- r_x \tau) g_x(\tau) \dtau.
\end{equation}
We can approximate the $r$--$\RR$ relationship by assuming that the generation-interval distribution is Gamma-distributed, and summarizing it using the mean generation interval $\bar{G}_x$ and the squared coefficient of variation $\kappa_x$ \citep{park2019practical}:
\begin{equation}
\RR_x \approx (1 + \kappa_x r_x \bar{G}_x)^{1/\kappa_x}.
\end{equation}
The Gamma assumption includes as a special case models that assume exponentially distributed generation intervals (when $\kappa=1$), corresponding to the SIR model \citep{anderson1991infectious}; various gamma assumptions and approximations are widely used in epidemic modeling, including for models of SARS-CoV-2 \citep{doi:10.1098/rsif.2020.0144}.
We use this framework to investigate how inferences about strength and speed of the variant depend on our assumptions about the underlying generation-interval distributions.
For simplicity we neglect differences in the squared coefficient of variation and assume $\kappa_{\mathrm{wt}} = \kappa_{\mathrm{var}} = \kappa$; instead, we focus on the effect of potential differences in the mean generation intervals.

\section{Inferring relative strength from relative speed}

Epidemic speed $r_x$ can often be estimated directly from case data during the exponential growth period \citep{mills2004transmissibility,nishiura2009transmission,ma2014estimating}---in theory both $r$ and $\RR$ relate to latent infections and therefore using case data can lead to biases due to delays in reporting \citep{goldstein2009reconstructing,gostic2020practical}.
However, estimating $r_x$ can be challenging when case counts are low (as when a new variant begins to spread) and is sensitive to temporal changes in testing patterns and intensity.
Studies of new SARS-CoV-2 variants have mostly focused on characterizing changes in the \emph{proportion} of a new variant \citep{switzerland2021variant, davies2021estimated, di2021impact, leung2021early, volz2021transmission,zhao2021}.
Assessing frequency dynamics has the advantage of being less sensitive to changes in testing and to other transient effects that would affect variants and wild type viruses similarly.
When incidence is changing exponentially ($i_x(t) = i_x(t_0) \exp(r_x t)$), the proportion of the new variant $p(t)$ follows a logistic growth curve:
\begin{align}
p(t) &= \frac{\iv(t_0) \exp(\rv t)}{\iw(t_0) \exp(\rw t) + \iv(t_0) \exp(\rv t)},
\\ &= \frac{1}{1 + \left(\iw(t_0)/\iv(t_0)\right) \exp(-\delta t)},
\end{align}
where the logistic growth rate $\delta = \rv - \rw$ corresponds to the relative speed of the epidemic.

\begin{figure}[!t]
\includegraphics[width=\textwidth]{relstrength.pdf}
\caption{
\textbf{Relative strength of the new variant conditional on observed relative speed under four epidemiological conditions.}
The relative strength of the new variant $\rho$ conditional on the speed of the wild type $\rw$, the ratio between the mean generation interval of the new variant $\Gv$ and that of the wild type $\Gw$, and the squared coefficient of variation in generation intervals $\kappa$.
The relative strength of the new variant $\rho$ is calculated using $\delta=0.1\pday$, $\Gw = 5\days$, and $\kappa = 1/5$.
Five epidemiological conditions assume $\rw=-0.15\pday$ ($\rw < \rv < 0$), $\rw=-0.1\pday$ ($\rw < \rv = 0$), $\rw=-0.05\pday$ ($\rw < 0 < \rv$), $\rw=0\pday$ ($0 = \rw < \rv$), and $\rw=0.05\pday$ ($0 < \rw < \rv$).
}
\label{fig:relstrength}
\end{figure}

We thus ask: what factors affect the relative strength $\rho = \Rv/\Rw$ of a new variant, conditional on an observed relative speed $\delta$?
Inference of $\rho$ from $\delta$ will depend on assumptions about the generation-interval distributions of both strains.
Given the mean generation interval of the variant $\Gv$ and the wild type $\Gw$, the relative strength $\rho = \Rv/\Rw$ under the gamma approximation \citep{park2019practical} is given by:
\begin{equation}
\rho = \left(\frac{1 + \kappa (\rw + \delta) \Gv}{1 + \kappa \rw \Gw}\right)^{1/\kappa}.
\end{equation}
Therefore, the relative strength $\rho$ depends not only on the relative speed $\delta$ and the generation-interval distributions but also on how fast the wild type is spreading in the population (\rw)---
some analyses have implicitly or explicitly neglected this factor by either assuming $\rw = 0$ \citep{switzerland2021variant} or $\kappa = 0$ \citep{davies2021estimated} (in the latter case $\hat{\rho} = \exp(\delta \bar{G})$).

Based on the observed relative growth rate of B.1.1.7 in the UK, we start by  taking the relative speed of the variant to be $\delta = 0.1/\textrm{day}$  \citep{davies2021estimated}; the mean generation interval of the wild type to be $\Gw = 5\days$ \citep{ferretti2020quantifying}; and the squared coefficient of variation of generation intervals to be $\kappa=0.2$ \citep{ferretti2020quantifying}.
We consider the spread of B.1.1.7 as an example but out qualitative conclusions should hold for other VoCs.
As noted above, we assume that the variant and the wild type have equal $\kappa$ throughout, and only consider differences in the mean.
We evaluate the estimates of relative strength $\rho$ across a wide range of $\kappa$ from 0 (fixed generation interval; delta distribution) to 1 (exponential distribution).
To further explore how inference depends on underlying epidemiological conditions (i.e., whether the incidence of the variant and the wild type are growing or shrinking), we consider five epidemiological conditions: (1) $\rw < \rv < 0$, (2) $\rw < \rv = 0$, (3) $\rw < 0 < \rv$, (4) $0 = \rw < \rv$, and (5) $0 < \rw < \rv$.

\fref{relstrength} illustrates how the relative strength depends on the shape of the generation-interval distributions and on underlying epidemiological conditions.
In general, longer mean generation interval of the new variant translates to higher values of $\rho$ (and vice versa), except when $\rv \leq 0$.
When $\rv = 0$ (and therefore $\rw < 0$), we always have $\Rv = 1$ and so the inferred value of $\rho$ is independent of the generation-interval distribution of the new variant.
When $\rv < 0$ (and therefore $\rw < 0$), we see that longer generation intervals decrease $\rho$ because longer generation intervals actually lead to slower decay (higher $r$).
Assuming a narrower distribution (lower $\kappa$) has qualitatively similar effects as assuming longer generation intervals because both reduce the amount of early transmission; 
therefore, narrower distributions lead to higher values of $\rho$ when $\rv > 0$ and lower values of $\rho$ when $\rv \leq 0$.
When $\rw < 0 < \rv$, inference of $\rho$ is relatively insensitive to values of $\kappa$.

We find that an increased speed of $\delta=0.1\pday$ for the variant is consistent with higher strength than the wild type across a wide range of epidemiological conditions \citep{switzerland2021variant, davies2021estimated, di2021impact, leung2021early, volz2021transmission,zhao2021}.
However, the magnitude of relative strength $\rho$ is sensitive to assumptions about generation intervals.
For realistic values of $\kappa$ (excluding 0 and 1), we find that the inferred relative strength $\rho$ ranges between 1.1--2.3.
Even if we restrict the mean generation interval of the variant to differ by 20\% and only consider the scenario under which variant is growing (bottom two panels of \fref{relstrength}), we find that $\rho$ ranges between 1.3--1.8, providing narrower but still large uncertainty.

\section{Inferring relative speed from relative strength}

We do not generally expect the relative speed $\delta$ to remain constant if other factors governing epidemic spread are changing.
Indeed, many biological mechanisms better translate to assuming a constant value of relative strength $\rho$ over changing conditions.
For example, if the proportion of the population susceptible declines, or the average contact rate changes, while other factors remain constant, the relative strength $\rho$ is expected to remain constant; 
in general, this will imply a change in relative speed $\delta$.

We thus investigate how $\delta$ is expected to change with \Rw if $\rho$ remains constant, and how this expectation changes with the ratio of the generation intervals. 
Once again, we rely on the gamma approximation to find the relative speed $\delta$ given the mean generation interval of the variant $\Gv$ and the wild type $\Gw$, :
\begin{equation}
\delta = \frac{(\rho \Rw)^{\kappa} - 1}{\kappa \Gv} - \frac{\Rw^{\kappa} - 1}{\kappa \Gw}.
\end{equation}
As our baseline scenario we assume $\rho = 1.61$, which is the value we obtain for $\delta=0.1\,\textrm{days}$, $\Gw = \Gv = 5\,\textrm{days}$, and $\kappa = 1/5$.
We evaluate $\delta$ across five different epidemiological conditions as before: (1) $\Rw < \Rv < 1$, (2) $\Rw < \Rv = 1$, (3) $\Rw < 1 < \Rv$, (4) $1 = \Rw < \Rv$, and (5) $1 < \Rw < \Rv$.

\begin{figure}[!th]
\includegraphics[width=\textwidth]{relspeed.pdf}
\caption{
\textbf{Relative speed of the new variant given assumed relative strength under four epidemiological conditions.}
The relative speed of the new variant $\delta$ given strength of the wild type $\Rw$, ratio between the mean generation interval of the new variant $\Gv$ and that of the wild type $\Gw$, and squared coefficient of variation in generation intervals $\kappa$.
Relative speed of the new variant $\delta$ is calculated using $\rho=1.61$, $\Gw = 5\days$, and $\kappa = 1/5$.
Five epidemiological conditions assume $\Rw=1/\rho^{3/2}$ ($\Rw < \Rv < 1$), $\Rw=1/\rho$ ($\Rw < \Rv = 1$), $\Rw=1/\rho^{1/2}$ ($\Rw < 1 < \Rv$), $\Rw=1$ ($1 = \Rw < \Rv$), and $\Rw=\rho^{1/2}$ ($0 < \Rw < \Rv$).
}
\label{fig:relspeed}
\end{figure}

In general, longer generation intervals lead to slower relative speed of the variant when the incidence of both strains is increasing (\fref{relspeed}, bottom panels) because slower growth of the variant reduces the differences in absolute speed.
When $\Rv=1$, the relative speed is insensitive to the generation-interval distribution of the variant because we always have $\rv=0$.
When $\Rv<1$, longer generation intervals of the variant lead to slower decay ($\rv$ closer to 0), and therefore, greater relative speed.
Once again, we see that assuming a narrower distribution (lower $\kappa$) has qualitatively similar effects as assuming a longer mean.

\fref{relspeed} also shows that when $\rho$ is fixed relative speed depends on underlying epidemiological conditions. 
For example, even when there are no differences in the generation-interval distributions ($\Gv=\Gw$, in this case), the relative speed $\delta$ can range between 0.08--0.11 when $\kappa=0.2$ and 0.06--0.14 when $\kappa=0.5$.
Therefore, characterizing the spread of variants assuming constant relative speed (e.g., by fitting a standard logistic growth curve) without considering how epidemiological conditions change over time should be avoided.

\section{Inferring relative strength from incidence data}

Instead of estimating relative strength from speed, one can estimate time-varying or instantaneous reproduction numbers $\RR(t)$ of the variant and the wild type from incidence data \citep{fraser2007estimating}, and directly compare their ratios;
such methods have been used in previous analyses of the B.1.1.7 strain by \cite{volz2021transmission}.
Assuming that the generation-interval distribution remains constant, the instantaneous reproduction numbers of the new variant and of the wild type can be estimated from their corresponding incidence curves---an approach popularized by \cite{cori2013new}:
\begin{equation}
\RR_x(t) = \frac{i_x(t)}{\int_0^\infty i_x(t-\tau) g_x(\tau) \dtau}.
\label{eq:rt}
\end{equation}
Under strength-like intervention measures that reduce transmission rates by a constant amount, we expect ratios between reproduction numbers to remain constant and correspond to the true relative strength: $\Rv(t)/\Rw(t) = \rho$.
However, if the assumed generation-interval distribution $\hat{g}(\tau)$ is different from the true distribution, then the ratio between the estimated reproduction numbers $\hat{\rho}(t) = \hat{\RR}_{\textrm{var}}(t)/\hat{\RR}_{\textrm{wt}}(t)$ may change, even if the true ratio does not.

Here, we investigate how misspecification of the generation-interval distribution of the variant affects our inference of relative strength from inference data under the assumption that the true generation-interval distribution of the wild type is known.
We use a two-strain renewal equation that assumes perfect cross-immunity to simulate three different scenarios (see Methods):  
(1) the wild type and the variant have the same generation-interval distributions (which match the known distribution, shown in \fref{Rtbias}A--C);
(2) the variant has a shorter mean generation interval (\fref{Rtbias}D--F); and
(3) the variant has a longer mean generation interval (\fref{Rtbias}G--I).
Then, we compare the estimated ratio $\hat{\rho}(t) = \hat{\RR}_{\textrm{var}}(t)/\hat{\RR}_{\textrm{wt}}(t)$ with the true ratio $\rho = \Rv(t)/\Rw(t)$.
In order to allow for changes in $\RR$ driven by the introduction and lifting of non-pharmaceutical interventions \citep{flaxman2020Rt}, we model $\Rw(t)$ using a step function (\fref{Rtbias}) and let $\Rv(t) = \rho \Rw(t)$.

\begin{figure}[!pht]
\includegraphics[width=\textwidth]{Rtbias.pdf}
\caption{
\textbf{Estimates of relative strength over time under different scenarios.}
(A, D, G) True (solid lines) and estimated (dashed lines) reproduction numbers of the new variant and the wild type over time.
(B, E, H) True (purple, solid) and estimated (orange, dashed) ratios between reproduction numbers of the new variant and the wild type over time.
(C, F, I) True (purple, solid) and estimated (orange, dashed) relationships between estimated reproduction numbers.
Blue dotted lines represent the regression lines of the estimated variant reproduction numbers against the estimated wild type reproduction numbers.
Gray lines represent the one-to-one line.
Top row: the assumed mean generation-interval (5 days) is equal to the mean generation-interval of the wild type and the variant.
Middle row: the assumed mean generation-interval (5 days) is equal to the mean generation-interval of the wild type but is longer than that of the variant (4 days).
Bottom row: the assumed mean generation-interval (5 days) is equal to the mean generation-interval of the wild type but is longer than that of the variant (6 days).
For all simulations, squared coefficient of variation in generation intervals is assumed to equal $\kappa = 1/5$.
}
\label{fig:Rtbias}
\end{figure}

When the assumed distribution matches the true distribution, the estimated reproduction numbers match the true values (\fref{Rtbias}A), and therefore the ratio remains constant (\fref{Rtbias}B).
However, if the generation-interval distribution of the variant differs from the assumed distribution, the ratio changes over time (\fref{Rtbias}E,H).
In particular, if the true generation intervals have a shorter mean than the assumed distribution, we over-estimate $\Rv(t)$ during the growth phase and under-estimate during the decay phase (and vice versa, \fref{Rtbias}D,G), which further translates to biases in the estimated relative strength (\fref{Rtbias}E,H).
Therefore, biases in the estimated relative strength will correlate with the strength of the variant, (i.e., whether $\Rv(t)$ is below or above 1).

In practice, estimates of instantaneous reproduction numbers $\RR(t)$ (and therefore, their ratios) can be noisy due to limited data availability or model assumptions;
instead, we might want to estimate a single value of relative strength $\rho$.
For example, we can estimate $\rho$ by plotting the estimated strength of the variant $\hat{\RR}_{\textrm{var}}(t)$ against the estimated strength of the wild type $\hat{\RR}_{\textrm{wt}}(t)$---as presented in Figure 2 of \cite{volz2021transmission}---and performing a linear regression (\fref{Rtbias}C,F,I).
When generation-interval distributions are correctly specified, we obtain a straight line with a slope of $\rho$ and intercept at zero (\fref{Rtbias}C).
However, when the assumed mean generation interval is longer than the that of the variant, we over-estimate the slope (and vice versa, \fref{Rtbias}F,I).
In theory, we expect the regression line to go through the origin, but when the assumed generation-interval distribution differs from the true distribution, this is not necessarily the case: biases in slopes due to mis-specified generation-interval distributions also lead to biases in intercepts.
This analysis suggests another method for assessing potential biases in the estimates of relative strength.

\section{Implications for intervention strategies}

While relative speed $\delta$ and strength $\rho$ are useful for characterizing the spread of the variant in an epidemiological context with a previously dominant wild type, \emph{absolute} speed $\rv$ and strength $\Rv$ of the variant are the main variables that determine the spread and conditions for control of the variant over the long term.
In particular, at any given point in epidemic, we can measure the speed of the variant $\rv$ (or infer $\rv$ from $\rw$ and $\delta$) and ask how much more intervention is required to control the spread of both strains (since $\Rv < 1$ implies $\Rw < 1$).
As a baseline scenario, we assume $\rw=0$ and $\delta=0.1\pday$ (and therefore $\rv=0.1\pday$), in which case additional intervention is required to reduce $\rv$ below 0 (or, equivalently, $\Rv$ below 1).

We consider two types of intervention:
an intervention of constant strength, which reduces transmission by a constant factor $\theta$ regardless of age of infection ($K_{\mathrm{post}}(\tau) = K_{\mathrm{pre}}(\tau)/\theta$); and an intervention of constant speed, which reduces transmission after infection by a constant rate $\phi$ ($K_{\mathrm{post}}(\tau) = K_{\mathrm{pre}}(\tau) \exp(-\phi \tau)$).
In this case, we can control the spread of the variant when $\theta > \Rv$ or $\phi > \rv$, respectively.
As a baseline scenario, we consider constant-strength and speed interventions that reduce $\Rv$ to 0.9 when both the variant and the wild type have equal mean generation intervals (\fref{strengthspeed}A--B).
While both interventions are equally effective on strength-scale (that is, $\Ry{post}=\int  K_{\mathrm{post}}(\tau) \dtau = 0.9$), they have different dynamical implications.
The constant-strength intervention affects transmission equally throughout the course of infection, whereas the constant-speed intervention has greater impact of transmission that occur later;
as a result, the constant-speed intervention reduces the post-intervention mean generation interval (\fref{strengthspeed}B) and leads to (slightly) faster exponential decay (therefore, lower $\ry{post}$).

\begin{figure}[!th]
\includegraphics[width=\textwidth]{control.pdf}
\caption{
\textbf{Effects of constant-strength and speed of interventions against the spread new variant.}
(A--B) Pre-intervention (black) and post-intervention kernel (colored) of the new variant under constant-strength (A) and speed (B) interventions assuming that the variant has equal generation intervals as the wild type.
(C--D) Pre-intervention (black) and post-intervention kernel (colored) of the new variant under constant-strength (C) and speed (D) interventions assuming that the variant has longer generation intervals as the wild type.
(E) Pre-intervention (black) and post-intervention strength conditional on the mean generation interval of the variant (colored).
(F) Pre-intervention (black) and post-intervention speed conditional on the mean generation interval of the variant (colored).
Strength and speed are calculated assuming$\rw=0\,\textrm{/days}$, $\delta=0.1\,\textrm{/days}$, $\rv=0\,\textrm{/days}$, $\Gy{wt}=5\,\textrm{days}$, and $\kappa=1/5$ for pre-intervention conditions.
Constant strength and speed interventions are modeled such that post intervention strength of the new variant, assuming equal mean generation intervals at 5 days, corresponds to 0.9.
}
\label{fig:strengthspeed}
\end{figure}

However, if the variant has longer generation intervals than the wild type (\fref{strengthspeed}C--D),  then the strength of the variant will be higher conditional on the observed speed (\fref{strengthspeed}E).
In this case, the same constant-strength intervention can fail to control the epidemic (i.e., $\Ry{post} > 1$; \fref{strengthspeed}E) because this intervention reduces the transmission by a constant amount regardless of age of infection (\fref{strengthspeed}C).
On the other hand, the same constant-speed intervention will prevent a larger proportion of transmission, leading to lower $\Ry{post}$ (\fref{strengthspeed}E), because it is more effective against late-stage transmission (\fref{strengthspeed}D).
The constant-speed intervention also has a greater impact on reducing the mean generation interval (\fref{strengthspeed}D).

The speed-based paradigm gives the same results about the control but provides important insights (\fref{strengthspeed}F).
The observed speed of the variant $\rv$ at a given moment is independent of our estimates of its mean generation interval.
Likewise, the post-intervention speed of the variant under the constant-speed intervention is also independent of the mean generation interval;
therefore, the constant-speed intervention will be always equally effective (on the speed scale) in preventing the spread.

The constant post-intervention speed can be understood by considering a speed-based decomposition of the kernel: $K_{\mathrm{pre}}(\tau) = \exp(\rv) b(\tau)$, where $b(\tau)$ is the initial backward generation-interval distribution and therefore integrates to 1 \citep{doi:10.1098/rspb.2020.1556}.
This decomposition allows us to re-write the post-intervention kernel (under the constant-speed intervention) as follows:
\begin{equation}
K_{\mathrm{post}}(\tau) = \exp(\rv-\phi) b(\tau).
\end{equation}
Since the post-intervention speed $\ry{post}$ is determined by the following implicit equation,
\begin{equation}
1 = \int_0^\infty \exp(-\ry{post} \tau) K_{\mathrm{post}}(\tau) \dtau = \int_0^\infty \exp(\rv-\phi-\ry{post}) b(\tau)  \dtau,
\end{equation}
we have $\ry{post} = \phi - \rv$. Therefore, if we can reduce infectiousness at a faster rate than the observed speed of spread (i.e., if the speed of intervention is faster than the epidemic speed, $\phi > \rv$), we can control the epidemic regardless of the underlying generation-interval distribution.

\section{Discussion}

We explored how the generation-interval distribution shapes the link between relative strength and speed of the new variant.
Longer generation intervals generally lead to higher relative strength for a given relative speed (and conversely, lower relative speed for a given relative strength); these can be reversed when incidence is decreasing.
Neglecting potential differences in the mean generation intervals between the variant and the wild type can bias estimates of the relative strength from incidence data;
these biases may be systematically assessed by considering whether apparent changes in relative strength over time are associated with underlying epidemiological dynamics of the variant and the wild type.
Finally, differences in generation intervals can further lead to different conclusions about the effectiveness of interventions:
if the variant has longer (respectively, shorter) generation intervals than the wild type, speed-like (strength-like) interventions, will be more relatively more effective than naive estimates would suggest.

As the new variants, in particular the B.1.1.7 variant, of SARS-CoV-2 are spreading and becoming dominant in many countries, it is clear that the variants are more transmissible (have higher strength) than the wild type \citep{switzerland2021variant, davies2021estimated, di2021impact, leung2021early, volz2021transmission,zhao2021}.
Our analysis supports estimates of a higher strength of the new variant bounding on a wide range of parameter ranges about epidemiological conditions and generation-interval distributions.
However, our analysis also shows that uncertainty in generation-interval distributions must be taken into account to obtain accurate estimates of the relative strength of variants.
If the variant has a longer infectious period \citep{kissler2021densely}, and therefore extended generation-interval distributions, current estimates may be underestimating the true relative strength.

There is currently a (minor) discrepancy in the estimates of relative strength of new variants, particularly for B.1.1.7.
Mathematical analyses have typically reported greater than 1.4 fold increase in reproduction number for the B.1.1.7 variant ($\rho > 1.4$) whereas an independent analysis of secondary attack rates from contact tracing data suggests 1.25--1.4 fold increase \citep{ukinvest}.
We can ask whether different estimates can be reconciled by generation intervals alone---for simplicity, we assume that the variant and the wild type have identical generation-interval distributions as other studies have.
For example, \cite{davies2021estimated} estimated $\rho=1.77$ assuming a fixed generation generation interval at 5.5 days from $\delta=0.104\pday$ in their regression analysis;
similarly, \cite{volz2021transmission} estimated $\rho=1.79$ by directly comparing the ratio between $\Rv$ and $\Rw$ assuming a mean generation interval of 6.5 days and a squared coefficient of variation of 0.38.
While \cite{volz2021transmission} assumed a longer mean generation interval (which typically increases $\rho$), their estimate is similar to \cite{davies2021estimated}, likely because the effect of a wide distribution (which typically reduces $\rho$) can cancel out with the effect of longer intervals---in fact, we obtain $\rho=1.81$ from $\delta=0.104\pday$, $\rw=0\pday$, $\Gv=\Gw=6.4\days$, and $\kappa=0.38$ under the gamma assumption, explaining the lack of differences in the estimates of $\rho$ despite the differences in the assumed generation-interval distributions.

However, a fixed generation generation interval assumed by \cite{davies2021estimated} is too narrow.
The mean generation interval of 6.5 days assumed by \cite{volz2021transmission} is also likely too long:
first, this estimate actually corresponds to the observed mean serial intervals from China in between January and February, 2020 \citep{bi2020epidemiology}, during which they were more likely to observe longer serial intervals due to initial epidemic growth \citep{park2021forward, ali2020serial}; and second, recently infected individuals may be less likely to transmit after symptom onset due to increased awareness of the pandemic compared to earlier on, leading to shorter generation intervals.
Instead, if we replace the assumed generation-interval distributions for \cite{volz2021transmission} with shorter generation interval estimates from Tianjin, China ($\Gw=2.57\days$ and $\kappa=1$; \cite{ganyani2020estimating}), we obtain $\rho=1.27$, which is consistent with the attack rate analysis \citep{ukinvest}.
We do not claim that these estimates of $\rho$ or $\Gw$ are more accurate;
instead, this back-of-the-envelope calculation highlights the importance of propagating uncertainty in the generation-interval distribution in assessing the relative strength and speed of SARS-CoV-2 variants.
Obtaining an accurate and precise estimate of $\rho$ will need to take into account changes in $\rho$ as well as underlying $\rw$.

We further used simulations to show how mis-specification of generation-interval distributions can bias the inference of relative strength from incidence data:
when the assumed generation-interval distribution does not match the true generation-interval distribution of the variant, estimates of absolute strength $\Rv(t)$ will be biased, causing the estimated relative strength to change over time.
In doing so, we assumed that the intervention will reduce transmission caused by the variant and the wild type by equal amounts, thereby preserving the relative strength over time;
however, this is only true under strength-like interventions, which is insensitive to age of infection, but not under speed-like interventions. 
As we demonstrated earlier, the effectiveness of speed-like interventions depend on the shape of the generation-interval distribution;
therefore, if the variant and the wild type have different generation intervals, speed-like interventions, such as contact tracing, can affect them differently.

We also assumed that the incidence of infection caused by the variant and the wild type is known.
However, estimating $\RR(t)$ from real data, such as case incidence, is often associated with several practical challenges, such as delays between infection and reporting and changes in testing patterns \citep{gostic2020practical}.
We have chosen to focus here on the underlying dynamical mechanisms that may impact inference.
Future studies may consider the development of methods based on mechanism outlined here to assess potential biases in their estimates of relative strength. 

In moving from mechanism to improved estimates, we argue that both perspectives (i.e., whether we hold relative strength or speed constant) are useful in understanding the dynamics of the new variant.
Early in the spread of an emerging variant, it is likely more convenient to fix the relative speed, given that speed can be directly observed.
As more information about the transmission and immunity profiles of the new variant becomes available, we advise fixing the relative strength and inferring the speed as this assumption better matches biological mechanisms for its higher strength.
Researchers should be mindful about which quantity they hold constant and how their assumptions lead to their conclusions \citep{doi:10.1098/rspb.2020.1556}.

Here, we lay out a few practical advice in analyzing the epidemiological dynamics of new variants for future studies.
First, models that assume a constant relative speed, such as standard logistic growth model, should be avoided for characterizing the growth of new variants---the relative speed changes over time depending on epidemiological conditions.
Second, the absolute strength and speed should not be neglected over their relative values, and both relative and absolute values should be reported together.
While the relative strength and speed are useful for describing the spread of new variants, the absolute values determine their spread and control.
In addition, as we showed in our analysis, it is critical to understand the underlying epidemiological conditions of the variant and the wild type in interpreting the relative strength and speed.
Finally, uncertainty in generation intervals should be carefully considered.

Even though SARS-CoV-2 has been spreading for more than a year, there is still considerable uncertainty in its generation intervals.
Observational studies typically focus on serial intervals (i.e., time between symptom onset of the infector and the infectee; \cite{svensson2007note}) because they are easier to measure \citep{griffin2020rapid}.
Furthermore, realized serial intervals are subject to dynamical biases that can be difficult to tease apart \citep{park2021forward}.
A few studies have tried to estimate the generation-interval distribution from serial intervals, with means ranging between 3--6 days and squared coefficients of variation ranging between 0.1--1 \citep{ferretti2020quantifying,Ferretti2020timing,ganyani2020estimating,knight2020estimating}; 
as we have shown, this uncertainty alone can generate large uncertainty in the inference of relative strength.
Furthermore, most estimates neglect dynamical biases in serial intervals caused by underlying epidemiological dynamics in their estimates of generation intervals.
Serial intervals also do not account for asymptomatic transmission, adding further uncertainty to inferences of speed and strength \citep{park2020time}.

Since the beginning of the SARS-CoV-2 pandemic, a few studies have emphasized the relevance of leveraging generation interval distributions to improve estimates of strength (e.g., \cite{doi:10.1098/rsif.2020.0144,park2021forward}) and the relative importance of modes of transmission (e.g., asymptomatic vs. symptomatic, \cite{park2020time}).
These same lessons should be considered when assessing the spread of variants in a partially susceptible population and improving efforts to control spread.
Future studies should prioritize detailed assessment of the generation intervals of SARS-CoV-2 and widespread variants, as well as consider how uncertainty in generational intervals might bias conclusions.

The spread of new SARS-CoV-2 variants and the replacement of previously dominant lineages represent ongoing challenges for controlling the SARS-CoV-2 pandemic.  
By explicitly considering epidemiological context and generation interval differences together, we have shown that improving estimates of the the relative duration of infectiousness at the individual scale may represent a pathway towards more effective interventions. 
Specifically, we have shown that speed-like interventions, such as contact tracing, will be more effective if variants have longer generation intervals.
Most intervention strategies throughout the current pandemic have focused on strength-like interventions \citep{flaxman2020Rt}, such as lock-downs, partly because pre-symptomatic transmission of SARS-CoV2 have limited the effectiveness of contact tracing efforts \citep{hellewell2020feasibility}.
However, given the possibility that new variants can have different infection characteristics, future studies should consider whether their transmission dynamics also differ (e.g., the amount of pre-symptomatic transmission) and evaluate intervention strategies accordingly.

\section{Methods}

\subsection{Two-strain renewal equation}

We use a two-strain renewal equation to simulate the spread of the variant and the wild type.
Ignoring birth and death, the incidence of infection caused by the variant $\iv$ and the wild type $\iw$ assuming perfect cross immunity is given by:
\begin{align}
\frac{dS}{dt} &= - \iv(t) - \iw(t)\\
\iv(t) &= \Rv(t) \int_{0}^\infty \iv(t-\tau) g_{\mathrm{var}}(\tau) \dtau\\
\iw(t) &= \Rw(t) \int_{0}^\infty \iw(t-\tau) g_{\mathrm{wt}}(\tau) \dtau
\end{align}
where $S$ represents the proportion of susceptible individuals.
We discretize the model at the time scale of 0.025 days as described in \citep{park2021forward} and simulated the model with $\iv(0) = 0.001$ and $\iw(0) = 0.1$.
We assume that $\Rw(t)$ remains at 2 for 30 days, immediately decreases and stays at 0.5 until day 60, and immediately increases back to 1.2.
Finally, we assume $\Rv(t) = \rho \Rw(t)$.
From simulated incidence curves between day 15 and 70, we estimate the instantaneous reproduction number using \eref{rt}.
We ignore incidence before day 15 to remove any potential transient effects.

\bibliography{newvariant.bib}

\end{document}
