\documentclass[12pt]{article}
\usepackage[top=1in,left=1in, right = 1in, footskip=1in]{geometry}

\usepackage{graphicx}
\usepackage{xspace}
%\usepackage{adjustbox}

\newcommand{\comment}{\showcomment}
%% \newcommand{\comment}{\nocomment}

\newcommand{\showcomment}[3]{\textcolor{#1}{\textbf{[#2: }\textsl{#3}\textbf{]}}}
\newcommand{\nocomment}[3]{}

\newcommand{\jd}[1]{\comment{cyan}{JD}{#1}}
\newcommand{\swp}[1]{\comment{magenta}{SWP}{#1}}
\newcommand{\bmb}[1]{\comment{blue}{BMB}{#1}}
\newcommand{\djde}[1]{\comment{red}{DJDE}{#1}}

\newcommand{\eref}[1]{Eq.~\ref{eq:#1}}
\newcommand{\fref}[1]{Fig.~\ref{fig:#1}}
\newcommand{\Fref}[1]{Fig.~\ref{fig:#1}}
\newcommand{\sref}[1]{Sec.~\ref{#1}}
\newcommand{\frange}[2]{Fig.~\ref{fig:#1}--\ref{fig:#2}}
\newcommand{\tref}[1]{Table~\ref{tab:#1}}
\newcommand{\tlab}[1]{\label{tab:#1}}
\newcommand{\seminar}{SE\mbox{$^m$}I\mbox{$^n$}R}

\usepackage{amsthm}
\usepackage{amsmath}
\usepackage{amssymb}
\usepackage{amsfonts}

\usepackage{lineno}
\linenumbers

\usepackage[pdfencoding=auto, psdextra]{hyperref}

\usepackage{natbib}
\bibliographystyle{unsrt}
\date{\today}

\usepackage{xspace}
\newcommand*{\ie}{i.e.\@\xspace}

\usepackage{color}

%% Consistent, changeable style for subscripts
\newcommand{\vvvar}{\mathrm{var}}
\newcommand{\wwwt}{\mathrm{wt}}

\newcommand{\rx}[1]{\ensuremath{{r}_{#1}}\xspace} 
\newcommand{\ry}[1]{\rx{\mathrm{#1}}} 
\newcommand{\rw}{\rx{\wwwt}}
\newcommand{\rv}{\rx{\vvvar}}

\newcommand{\Rx}[1]{\ensuremath{{\mathcal R}_{#1}}\xspace} 
\newcommand{\Ry}[1]{\Rx{\mathrm{#1}}}
\newcommand{\Ro}{\Rx{0}}
\newcommand{\RR}{\ensuremath{{\mathcal R}}\xspace}
\newcommand{\Rw}{\Rx{\wwwt}}
\newcommand{\Rv}{\Rx{\vvvar}}

\newcommand{\days}{\ensuremath{\, \textrm{day}}}
\newcommand{\pday}{\ensuremath{/\textrm{day}}}
\newcommand{\dd}[1]{\ensuremath{\, \mathrm{d}#1}}
\newcommand{\dtau}{\dd{\tau}}
\newcommand{\dx}{\dd{x}}
\newcommand{\dsigma}{\dd{\sigma}}

\newcommand{\ix}[1]{\ensuremath{{i}_{#1}}\xspace} 
\newcommand{\iy}[1]{\ix{\mathrm{#1}}}

\newcommand{\Gx}[1]{\ensuremath{{\bar G}_{#1}}\xspace} 
\newcommand{\Gy}[1]{\Gx{\mathrm{#1}}}

\newcommand{\tsub}[2]{#1_{{\textrm{\tiny #2}}}}
\newcommand{\tstart}{\ensuremath{\tsub{t}{start}}\xspace}
\newcommand{\tend}{\ensuremath{\tsub{t}{end}}\xspace}

\newcommand{\betaeff}{\ensuremath{\tsub{\beta}{eff}}\xspace}
\newcommand{\Keff}{\ensuremath{\tsub{K}{eff}}\xspace}

\newcommand{\pt}{p} %% primary time
\newcommand{\st}{s} %% secondary time

\newcommand{\psize}{{\mathcal P}} %% primary cohort size
\newcommand{\ssize}{{\mathcal S}} %% secondary cohort size

\newcommand{\gtime}{\sigma} %% generation interval
\newcommand{\gdist}{g} %% generation-interval distribution

\newcommand{\geff}{g_{\textrm{eff}}} %% generation-interval distribution

\newcommand{\total}{{\mathcal T}} %% total number of serial intervals

\newcommand{\PP}{{\mathcal P}}
\newcommand{\II}{{\mathcal I}}

\begin{document}

\begin{flushleft}{
	\Large
	\textbf\newline{
		Roles of generation-interval distributions in characterizing relative epidemic strength and speed of new SARS-CoV-2 variants and their control
	}
}
\end{flushleft}

\section*{Abstract}

\section{Introduction}

Since the emergence of the new SARS-CoV-2 variant of concern (VoC), estimating its epidemic strength and speed remains one of the key questions in controlling its spread \citep{switzerland2021variant, davies2021estimated, di2021impact, leung2021early, volz2021transmission,zhao2021}.
Epidemic strength, characterized by the reproduction number $\RR$, provides information about the final size of the epidemic as well as the amount of intervention required to eliminate the disease \citep{anderson1991infectious}.
Epidemic speed, characterized by the growth rate $r$ provides information about the population-level time scale of the epidemic and, potentially provides information on when the new variant will replace the existing variant as the dominant strain.
Strength and speed are linked by generation intervals---defined as time between infection and transmission---which characterizes individual-level time scale of the epidemic \citep{roberts2007model,svensson2007note,wallinga2007generation}.

Many studies have tried to estimate the strength of the novel variant from the observed speed \citep{davies2021estimated, leung2021early, volz2021transmission,zhao2021}.
In contrast, some studies have assumed a value of the strength of the novel variant and tried to predict future spread (therefore, the speed) of the variant \citep{davies2021estimated}.
While both approaches are reasonable in principle, holding different quantities constant (i.e., strength or speed) can lead to different conclusions, in practice, about the spread of the disease, and therefore, its control.
The reason is that inference of speed from strength and strength from speed depends on assumptions about the generation-interval distribution \citep{roberts2007model,svensson2007note,wallinga2007generation,dushoff2020speed}.
For example, when we assume a fixed value of epidemic strength, longer generation intervals lead to faster epidemic (lower $r$), making the epidemic seem easier to control.
When we assume a fixed value of epidemic speed, longer generation intervals lead to stronger epidemic (higher $\RR$), making the epidemic seem more difficult to control.

In addition, most studies have assumed that the generation-interval distributions do not differ between the existing strain and the variant, but recent evidence suggests that the new variant may have a longer duration of infection: 13.3 days (90\% CI: 10.1--16.5) for the new variant and 8.2 days (90\% CI: 6.5--9.7) for the wild type \citep{kissler2021densely}.
This implies that the mean generation interval of the new variant is likely to be longer than that of the wild type.
Other studies have considered the possibility that the faster growth rate of the new variant may be driven, in part, by \emph{shorter} generation intervals \citep{davies2021estimated,volz2021transmission}.
If the generation-interval distribution of the new variant is different from the generation-interval distribution of the wild type, current estimates of the strength of the new variants may be biased.
More information is clearly needed, however. 
And the story may be complicated: generation intervals depend on many other biological and behavioral factors, including for example self-isolation after symptom onset leading to shorter generation intervals.

Here, we explore how different assumptions about the relative generation time of the new variant affect estimates of its strength $\RR$ and speed $r$.
We do this by exploring the relationship between relative strength (the ratio $\Rv/\Rw$) and relative speed (the difference $\rv-\rw$).
We find that neglecting differences in the generation-interval distributions can lead to biased estimates, which, in turn, influence mitigation priorities.
\jd{Just thoughts. I know we haven't introduced this terminology yet, and I know that you're going to use different terminology below. The point is that the old version was clunky and I'm trying to sharpen it. We can keep working.}

\section{Renewal equation framework}

We use the renewal equation framework to characterize the spread of two pathogen strains---in this case, the wild type SARS-CoV-2 virus and a focal variant of concern (e.g., B.1.1.7).
Neglecting the (relatively slow) rate of new mutations, the current incidence of infection $i_x(t)$ caused by the wild type (``wt'') and the variant (``var'') can be expressed in terms of their previous incidence $i_x(t-\tau)$ and the rate at which secondary cases are generated at time $t$ by individuals infected $\tau$ time units ago $K_x(t, \tau)$:
\begin{equation}
i_x(t) = \int_0^\infty i_x(t-\tau) K_x(t, \tau) \dtau.
\end{equation}
This framework provides a flexible way of modeling disease dynamics and generalizes compartmental models, such as the SEIR model \citep{heesterbeek1996concept, diekmann2000mathematical, roberts2004modelling, aldis2005integral, roberts2007model, champredon2018equivalence}.

The integral of the kernel $\RR_x(t) = \int K_x(t, \tau) \dtau$ is referred to as the instantaneous reproduction number \citep{fraser2007estimating}.
This is a particular kind of weighted average which describes the average infectiousness of previously infected individuals at time $t$---in particular, it is not weighted by the actual number of infected individuals present, but rather by their relative infectiousness at time $t$.
The normalized kernel $g_x(t, \tau) = K_x(t, \tau)/\RR_x(t)$---which we refer to as the instantaneous generation-interval distribution---describes their relative contribution to current incidence $i_x(t)$ and provides information about the time scale of disease transmission.
Both the reproduction number and the generation-interval distribution can depend on several factors, including intrinsic infectiousness of an infected individual, non-pharmaceutical interventions, awareness-driven behavior, and population-level susceptibility \citep{fraser2007estimating}.

Over a short period of time, we can assume that epidemiological conditions remain roughly constant: $\RR_x(t) \approx \RR_x$ and $g_x(t, \tau) \approx g_x(\tau)$.
In this case, the incidence of each strain will exhibit exponential growth (or decay) at rate $r_x$, satisfying the Euler-Lotka equation \citep{wallinga2007generation}:
\begin{equation}
\frac{1}{\RR_x} = \int_0^\infty \exp(- r_x \tau) g_x(\tau) \dtau.
\end{equation}
We can approximate the $r$--$\RR$ relationship by assuming that the generation-interval distribution is approximately gamma-distributed, and summarizing it using the mean generation interval $\bar{G}_x$ and the squared coefficient of variation $\kappa_x$ \citep{park2019practical}:
\begin{equation}
\RR_x \approx (1 + \kappa_x r_x \bar{G}_x)^{1/\kappa_x}.
\end{equation}
The gamma assumption covers models that assume exponentially distributed generation intervals (when $\kappa=1$), corresponding to the SIR model \citep{anderson1991infectious}; various gamma assumptions and approximations are widely used in epidemic modeling, including for models of SARS-CoV-2 \citep{doi:10.1098/rsif.2020.0144}.
We use this framework to investigate how inferences about strength and speed of the variant depend on our assumptions about the underlying generation-interval distributions.
For simplicity we neglect differences in the squared coefficient of variation and assume $\kappa_{\mathrm{wt}} = \kappa_{\mathrm{var}} = \kappa$; instead, we focus on the effect of potential differences in the mean generation intervals.

\section{Inferring relative strength from relative speed}

Epidemic speed $r_x$ can be often estimated directly from case data \citep{mills2004transmissibility,nishiura2009transmission,ma2014estimating};
this can be challenging when case counts are low (as when a new variant begins to spread) and is sensitive to temporal changes in testing patterns and intensity.
Studies of new SARS-CoV-2 variants have mostly focused on characterizing changes in \emph{proportion} of a new variant \citep{switzerland2021variant, davies2021estimated, di2021impact, leung2021early, volz2021transmission,zhao2021}.
This has the advantage of being less sensitive to changes in testing and to other transient effects that might affect variants and wild type viruses similarly.
When incidence is changing exponentially ($i_x(t) = i_x(t_0) \exp(r_x t)$), the proportion of the new variant $p(t)$ follows a logistic equation:
\begin{align}
p(t) &= \frac{\iy{var}(t_0) \exp(\rv t)}{\iy{wt}(t_0) \exp(\rw t) + \iy{var}(t_0) \exp(\rv t)},
\\ &= \frac{1}{1 + \left(\iy{wt}(t_0)/\iy{var}(t_0)\right) \exp(-\delta t)},
\end{align}
where the logistic growth rate $\delta = \rv - \rw$ corresponds to the relative speed of the epidemic.

\begin{figure}[!th]
\includegraphics[width=\textwidth]{relstrength.pdf}
\caption{
\textbf{Relative strength of the new variant given observed relative speed.}
(A) Estimated relative strength of the new variant $\hat{\rho}$ given speed of the wild type
$\rw$ assuming identical generation-interval distributions.
(B) True relative strength of the new variant $\rho$ given speed of the wild type $\rw$ and ratio between the mean generation interval of the new variant $\bar{G}_v$ and that of the wild type $\bar{G}_w$.
(C) Ratio between the true and the estimated relative strength of the new variant $\rho/\hat{\rho}$.
True relative strength of the new variant $\rho$ is calculated using $\delta=0.1\pday$, $\bar{G}_w = 5\days$, and $\kappa = 1/5$.
}
\label{fig:relstrength}
\end{figure}

We thus ask: what factors affect the relative strength $\rho = \Rv/\Rw$ of a new variant, given an observed relative speed $\delta$?
This inference will depend in general on assumptions about the growth rate of the wild type and on the generation-interval distributions of both variants.
These have generally been assumed to be the same, but we will also consider the possibility that they are different.

As our baseline scenario, we take the relative speed of the variant to be $\delta = 0.1/\textrm{day}$ \citep{davies2021estimated}; the mean generation interval of the wild type to be $\bar{G}_w = 5\days$ \citep{ferretti2020quantifying}; and the squared coefficient of variation of generation intervals to be $\kappa=0.2$ \citep{ferretti2020quantifying}.
For simplicity, we assume that the variant and the wild type have equal $\kappa$ throughout, and only consider differences in the mean.
\jd{This is awkward; let's skip for now:} As we expect there to be greater uncertainty in $\kappa$ than in $\bar{G}$, we evaluate the estimates of relative strength $\rho$ across a wide range of $\kappa$, ranging from 0 (delta distribution) to 1 (exponential distribution).

When the variant and wild type have the same generation-interval distributions, $\Gy{var} = \Gy{wt} = \bar{G}$, 
the estimate of the relative strength is given by \citep{park2019practical}:
\begin{equation}
\hat{\rho} = \left(1 + \frac{\kappa \delta \bar{G}}{1 + \kappa \rw \bar{G}}\right)^{1/\kappa}.
\label{eq:rhoGbar}
\end{equation}
Therefore, estimates of relative strength $\rho$ depend not only on the relative speed $\delta$ but also on how fast the wild type is spreading in the population \rw---a few analyses have implicitly or explicitly neglected this factor by either assuming $\rw = 0$ \citep{switzerland2021variant} or $\kappa = 0$ \citep{davies2021estimated} (in the latter case $\hat{\rho} = \exp(\delta \bar{G})$).

\eref{rhoGbar} shows that, for realistic (non-zero) values of $\kappa$, estimates of the strength $\rho$ go down as estimates of the growth rate $\rw$ go up. 
\jd{I like effective dispersion, but it does no good to just mention it once. Leave it out for now?}
The relative strength also depends on the \emph{effective dispersion} in generation intervals (i.e., the amount of variability in generation intervals, as measured by $\kappa$)---using narrower distributions makes $\hat{\rho}$ less sensitive to $\rw$.
These effects are illustrated in \fref{relstrength}A.

However, if the true mean generation interval of the new variant is different from that of the wild type ($\bar{G}_v \neq \bar{G}_w$), the true relative strength $\rho = \Rv/\Rw$ is given by:
\begin{equation}
\rho = \left(\frac{1 + \kappa (\rw + \delta) \bar{G}_v}{1 + \kappa \rw \bar{G}_w}\right)^{1/\kappa}.
\end{equation}
In this case, longer (or shorter) mean generation interval of the new variant generally translates to higher (or lower) values of $\rho$---\fref{relstrength}B illustrates this effect.
However, this direction can change when both the new variant and the wild type are decaying exponentially (see area below $\rv = 0$ line in \fref{relstrength}B)---for a given rate of decline, longer generation intervals lead to a \emph{lower} value of strength (and therefore lower relative strength $\rho$).

Therefore, neglecting potential differences in generation intervals can lead to mis-estimation of the relative strength and the variant, with implications for forecasting and planning.
In \fref{relstrength}C, we quantify the bias that arises by incorrectly assuming generation intervals are the same using the ratio between the true value $\rho$ and the estimated value $\hat{\rho}$, which assumes equal generation-interval distributions.
Across the ranges of parameters that we consider, we find that up to 1.4 fold (either upward or downward) adjustments to current estimates may be required.

\section{Inferring relative speed from relative strength}

We do not generally expect the relative speed $\delta$ to be remain constant if other factors governing epidemic spread are changing.
Indeed, many biological mechanisms better translate to assuming a constant value of relative strength $\rho$ over changing conditions.
For example, if the proportion of the population susceptible declines, or the average contact rate changes, while other factors remain constant, the relative strength $\rho$ is expected to remain relatively constant; in general, this will imply a change in relative speed $\delta$.

We thus investigate how $\delta$ is expected to change with \Rw, if $\rho$ remains constant, and how this expectation changes with the ratio of the generation intervals. As our baseline scenario we assume $\rho = 1.61$, which is the value we obtain for $\delta=0.1\,\textrm{days}$, $\bar{G}_w = \bar{G}_v = 5\,\textrm{days}$, and $\kappa = 1/5$.

\begin{figure}[!th]
\includegraphics[width=\textwidth]{relspeed.pdf}
\caption{
\textbf{Relative speed of the new variant given assumed relative strength.}
(A) Estimated relative speed of the new variant $\hat{\delta}$ given strength of the wild type $\Rw$ assuming identical generation-interval distributions.
(B) True relative strength of the new variant $\delta$ given strength of the wild type $\Rw$ and ratio between the mean generation interval of the new variant $\bar{G}_v$ and that of the wild type $\bar{G}_w$..
(C) Difference between the true and the estimated relative speed of the new variant $\delta - \hat{\delta}$.
True relative speed of the new variant $\delta$ is calculated using $\rho=1.61$, $\bar{G}_w = 5\,\textrm{days}$, and $\kappa=1/5$.
}
\label{fig:relspeed}
\end{figure}

As above, we first ignore possible differences between the generation-interval distributions of the wild type and the new variant and estimate $\delta$:
\begin{equation}
\hat{\delta} = \frac{(\rho \Rw)^{\kappa} - \Rw^{\kappa}}{\kappa \bar{G}_w}.
\end{equation}
Here, we see that the estimated $\hat{\delta}$ increases as the underlying strength of the wild type $\Rw$ increases (\fref{relspeed}A).
In other words, we expect the relative speed of the new variant to change as epidemiological conditions change, suggesting that a logistic model that assumes constant growth rate may not be appropriate for modeling the proportion of the new variant.
A narrower distribution causes $\hat{\delta}$ to be less sensitive to $\Rw$;
when $\kappa = 0$ (delta-distributed generation intervals), $\hat{\delta}$ remains constant.

When we allow the mean generation interval of the new variant to be different ($\bar{G}_v \neq \bar{G}_w$), the true $\rho$--$\delta$ relationship under gamma approximation is given by \citep{park2019practical}:
\begin{equation}
\delta = \frac{(\rho \Rw)^{\kappa} - 1}{\kappa \bar{G}_v} - \frac{\Rw^{\kappa} - 1}{\kappa \bar{G}_w}.
\end{equation}
In this case, $\delta$ generally decreases as the mean generation interval of the new variant increases, except when both $\Rv < 1$ and $\Rw < 1$ (\fref{relspeed}B).
In some extreme cases---when $\bar{G}_v$ and $\Rv$ are very low or when $\bar{G}_v$ and $\Rv$ are very large---we have $\delta < 0$. This is a scenario that does not seem to have occurred for B.1.1.7, where fast replacement has been observed under a wide variety of conditions, perhaps suggesting that any differences in generation interval for this variant are not too big.
Finally, assuming equal generation intervals generally leads to both under- and over-estimation of $\delta$ (\fref{relspeed}C).

\section{Inferring relative strength from incidence data}

In practice, epidemiological conditions change over time due to intervention measures and susceptible depletion.
Such changes can be characterized by ``time-varying'' or ``instantaneous'' reproduction numbers, $\RR(t)$ \citep{fraser2007estimating}.
Assuming that the generation-interval distribution does not change over time, the instantaneous reproduction numbers of the new variant and of the wild type can be estimated from their corresponding incidence curves---an approach popularized by \cite{cori2013new}:
\begin{equation}
\RR_x(t) = \frac{i_x(t)}{\int_0^\infty i_x(t-\tau) g_x(\tau) \dtau}.
\end{equation}
Under strength-like intervention measures that reduce transmission rates by a constant amount, we expect ratios between true reproduction numbers to remain constant and correspond to the true relative strength: $\Rv(t)/\Rw(t) = \rho$.
However, if the assumed generation-interval distribution $\hat{g}(\tau)$ is different from the true distribution, then the ratio between the estimated reproduction numbers $\hat{\rho}(t) = \hat{\RR}_{\textrm{var}}(t)/\hat{\RR}_{\textrm{wt}}(t)$ may change, even if the true ratio does not.

Here, we investigate under the assumption that the true generation-interval distribution of the wild type is known, and use a two-strain renewal equation which assumes perfect cross-immunity to simulate three different scenarios:  
(1) the wild type and the variant have the same generation-interval distributions (which match the known distribution, shown in \fref{Rtbias}A--C);
(2) the variant has a shorter mean generation interval (\fref{Rtbias}D--F); and
(3) the variant has a longter mean generation interval (\fref{Rtbias}G--I).
Then, we compare the estimated ratio $\hat{\rho}(t) = \hat{\RR}_{\textrm{var}}(t)/\hat{\RR}_{\textrm{wt}}(t)$ with the true ratio $\rho = \Rv(t)/\Rw(t)$.
In order to reflect smooth changes in $\RR$ driven by non-pharmaceutical interventions and behaviour changes, we model $\Rw(t)$ phenomenologically and let $\Rv(t) = \rho \Rw(t)$.

\begin{figure}[!th]
\includegraphics[width=\textwidth]{Rtbias.pdf}
\caption{
\textbf{Estimates of relative strength over time under different scenarios.}
(A, D, G) True (solid lines) and estimated (dashed lines) reproduction numbers of the new variant and the wild type over time.
(B, E, H) True (purple, solid) and estimated (orange, dashed) ratios between reproduction numbers of the new variant and the wild type over time.
(C, F, I) True (purple, solid) and estimated (orange, dashed) relationships between estimated reproduction numbers on a log-log scale.
Top row: the assumed mean generation-interval (5 days) is equal to the mean generation-interval of the wild type and the variant.
Middle row: the assumed mean generation-interval (5 days) is equal to the mean generation-interval of the wild type but is longer than that of the variant (4 days).
Bottom row: the assumed mean generation-interval (5 days) is equal to the mean generation-interval of the wild type but is longer than that of the variant (6 days).
For all simulations, squared coefficient of variation in generation intervals is assumed to equal $\kappa = 1/5$.
We assume $\Rw(t)=2.5$ before $t=15\,\textrm{days}$ to reflect unmitigated spread and let $\Rw(t)=2.5 (1+0.7\cos((t-15)/10))/1.7$ after to model smooth introduction and lifting of non-pharmaceutical interventions.
}
\label{fig:Rtbias}
\end{figure}

When the assumed distribution matches the true distributions, the estimated reproduction numbers match the true values (\fref{Rtbias}A), and therefore the ratio remains constant (\fref{Rtbias}B).
However, if the generation-interval distribution of the variant is different from the assumed distribution, the ratio changes over time (\fref{Rtbias}F,I).
In particular, if the true generation intervals have a shorter (or longer) mean than the assumed distribution, we over-estimate (or under-estimate) $\Rv(t)$ during the growth phase and under-estimate (or over-estimate) $ $ during the decay phase (\fref{Rtbias}D,G), which further translates to biases in the estimated relative strength (\fref{Rtbias}F,I).

We can assess these biases by plotting the estimated strength of the variant $\hat{\RR}_{\textrm{var}}(t)$ against the estimated strength of the wild type $\hat{\RR}_{\textrm{wt}}(t)$ on a log-log scale (\fref{Rtbias}C,F,I).
When generation-interval distributions are correctly specified, we obtain a straight line with a slope of 1. 
In this case, the intercept corresponds to the logarithm of the relative strength $\log(\rho)$ (\fref{Rtbias}C).
When the assumed mean generation interval is longer (or shorter) than the that of the variant, we over-estimate (or under-estimate) the slope (\fref{Rtbias}H,L) on a log-log plot.
Smooth changes in $\RR$ further generate circular patterns but biases in slopes are clear.

\section{Implications for intervention strategies}

While the relative speed need not stay constant over the course of an epidemic, at any given point in epidemic, we can measure the speed of the new variant and the wild type and ask how much more intervention is required to control the spread of both strains (relative to measured conditions).
As a baseline scenario, we assume $\rw=0\,\textrm{/days}$ and $\delta=0.1\,\textrm{/days}$ (and therefore $\rv=0\,\textrm{/days}$).
In this case, the wild type is already under control, whereas the variant requires additional intervention measures to stop growing.

Here, we consider two types of interventions:
a strength-like intervention which affects the kernel by a constant amount $\theta$ regardless of when they were infected ($K_{\mathrm{post}}(\tau) = K_{\mathrm{pre}}(\tau)/\theta$) and a speed-like intervention which affect the kernel at a constant rate $\phi$ ($K_{\mathrm{post}}(\tau) = K_{\mathrm{pre}}(\tau) \exp(-\phi \tau)$).
In this case, it is clear that we can control the spread of the variant when either $\theta > \Rv$ or $\phi > \rv$---these two quantities ($\theta$ and $\phi$) are also referred to as strength and speed of intervention, respectively.
In general, we can calculate the strength ($\theta = \Ry{pre}/\Ry{post}$) and speed ($\phi = \ry{pre} - \ry{post}$) of intervention, both providing equivalent measures for effectiveness of intervention, for any intervention by comparing epidemiological conditions before and after the intervention \citep{doi:10.1098/rspb.2020.1556}---these quantities, by definition, must be greater than pre-intervention strength and speed of the epidemic to control the spread.
We compare how potential differences in generation-interval distributions affect the control of the spread of the new variant using this strength-and-speed framework.

\begin{figure}[!th]
\includegraphics[width=\textwidth]{control.pdf}
\caption{
\textbf{Strength and speed of intervention against the new variant.}
(A) Strength of epidemic (black, solid), constant strength intervention (orange, dashed), and constant speed intervention (blue, dotted).
(B) Speed of epidemic (black, solid), constant strength intervention (orange, dashed), and constant speed intervention (blue, dotted).
Strength and speed are calculated assuming $\rw=0\,\textrm{/days}$, $\delta=0.1\,\textrm{/days}$, $\rv=0\,\textrm{/days}$, $\Gy{wt}=5\,\textrm{days}$, and $\kappa=1/5$.
Constant strength and speed interventions are modeled such that post intervention strength of the new variant, assuming equal mean generation intervals at 5 days, corresponds to 0.9.
}
\label{fig:strengthspeed}
\end{figure}

\fref{strengthspeed} compares the effectiveness of strength-based and speed-based interventions across a wide range of assumptions about generation-interval distributions.
We begin by assuming that the variant and the wild type have equal mean generation intervals and model interventions such that they both reduce $\Rv$ to 0.9 after intervention (\fref{strengthspeed}A).
If the mean generation interval of the variant is longer than that of the wild type, then the strength of the variant will be higher;
since the strength of the strength-like intervention is insensitive to the underlying generation-interval distribution, we may fail to control the spread of the new variant.
The strength of the speed-like intervention, however, depends on the new variant:
since speed-like interventions, such as contact tracing, have disproportionate effects on preventing later transmission, they are more effective (therefore stronger) when the mean generation interval of the new variant is longer.

The speed-based paradigm gives same results about the control but provides important insights (\fref{strengthspeed}B).
The observed speed of the variant at any given moment does not depend on our estimates of its mean generation interval.
Therefore, if we can prevent infections at a faster rate than the observed speed of spread (i.e., if the speed of intervention is faster than the epidemic speed), we can control the epidemic---this explains why a constant-speed intervention can prevent the spread of the variant regardless of its mean generation interval.
On the other hand, differences in mean generation intervals between the variant and the wild type imply different post-intervention epidemic speed (therefore, intervention speed) under a constant-strength intervention, explaining why it might not be effective if the variant has longer generation intervals.

\section{Discussion}

Here, we explored how the generation-interval distribution shapes the link between relative strength and speed of the new variant.
We define relative strength as the ratio between reproduction numbers ($\rho=\Rv/\Rw$) and relative speed as the difference between growth rates ($\delta=\rv-\rw$) of the variant and the wild type.
When the generation-interval distributions are assumed to be identical, we find that estimates of $\hat{\rho}$ decreases with $\rw$ whereas $\hat{\delta}$ increase with $\Rw$.
When we account for possible differences in generation-interval distributions, these relationships can change directions and further lead to biases in these estimates.
We show how these biases might be assessed in practice.
We also discuss that differences in generation intervals lead to different conclusions about the effectiveness of interventions:
if the variant has longer or shorter generation intervals than the wild type, we should prioritize speed- or strength-like interventions, respectively.

We argue that both perspectives (i.e., whether we hold relative strength or speed constant) are useful in understanding the dynamics of the new variant.
Early in the spread, we typically want to fix the relative speed because we can directly observe it.
As more information about the transmission and immunity profiles of the new variant becomes available, it is better to fix the relative strength as it better matches biological mechanisms for its higher strength.
We suggest researchers to be mindful about which quantity they hold constant and how their assumptions lead to their conclusions.

Our simple framework further reveals counterintuitive, but important, insights on how to control the spread of both wild type and variants.
In general, one would expect that stronger intervention strategies are required in order to slow down the replacement of the original strain by the new variant.
However, this is only true if both have the same mean generation intervals (\fref{relspeed}B).
If the new variant has longer generation intervals than the original strain, imposing strong intervention (therefore reducing both $\Rw$ and $\Rv$ by a constant amount) can result in a higher relative speed $\delta$, and therefore faster replacement of the original strain (\fref{relspeed}B).
This is not necessarily a bad outcome as the absolute speed ($\rv$) of the new variant still decreases as $\Rv$ decreases.
These effects should be taken into consideration for accurate projection of the spread of the new variant.

Even though SARS-CoV-2 has been spreading for more than a year, there is still considerable uncertainty in its generation intervals.
Observational studies typically focus on serial intervals (i.e., time between symptom onset of the infector and the infectee; \cite{svensson2007note}) because they are easier to measure \citep{griffin2020rapid}, but they are subject to dynamical biases that can be difficult to tease apart \citep{park2021forward}.
A few studies have tried to estimate the generation-interval distribution from serial intervals with means ranging between 3--6 days and squared coefficient of variations ranging between 0.1--0.6 \citep{ferretti2020quantifying,Ferretti2020timing,ganyani2020estimating,knight2020estimating}; 
however, most estimates neglect dynamical biases in serial intervals.
In addition, serial intervals do not account for asymptomatic transmission, adding further uncertainty to inferences of speed and strength \citep{park2020time}.

Since the beginning of the SARS-CoV-2 pandemic, a few studies have emphasized  the relevance of leveraging generation interval distributions to improve estimates of strength (e.g., \cite{doi:10.1098/rsif.2020.0144}) and the relative importance of modes of transmission (e.g., asymptomatic vs. symptomatic, \cite{park2020time}).
These same lessons have relevance for assessing the spread of variants in a partially susceptible population and improving efforts to control spread.
Future studies should prioritize detailed assessment of the relative generation intervals of SARS-CoV-2 and widespread variants, as well as evaluate how uncertainty in assumptions about generational intervals might bias conclusions.

\bibliography{newvariant.bib}

\end{document}
