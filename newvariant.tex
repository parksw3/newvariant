\documentclass[12pt]{article}
\usepackage[top=1in,left=1in, right = 1in, footskip=1in]{geometry}

\usepackage{graphicx}
\usepackage{xspace}
%\usepackage{adjustbox}

\newcommand{\comment}{\showcomment}
%% \newcommand{\comment}{\nocomment}

\newcommand{\showcomment}[3]{\textcolor{#1}{\textbf{[#2: }\textsl{#3}\textbf{]}}}
\newcommand{\nocomment}[3]{}

\newcommand{\jd}[1]{\comment{cyan}{JD}{#1}}
\newcommand{\swp}[1]{\comment{magenta}{SWP}{#1}}
\newcommand{\bmb}[1]{\comment{blue}{BMB}{#1}}
\newcommand{\djde}[1]{\comment{red}{DJDE}{#1}}

\newcommand{\eref}[1]{Eq.~\ref{eq:#1}}
\newcommand{\fref}[1]{Fig.~\ref{fig:#1}}
\newcommand{\Fref}[1]{Fig.~\ref{fig:#1}}
\newcommand{\sref}[1]{Sec.~\ref{#1}}
\newcommand{\frange}[2]{Fig.~\ref{fig:#1}--\ref{fig:#2}}
\newcommand{\tref}[1]{Table~\ref{tab:#1}}
\newcommand{\tlab}[1]{\label{tab:#1}}
\newcommand{\seminar}{SE\mbox{$^m$}I\mbox{$^n$}R}

\usepackage{amsthm}
\usepackage{amsmath}
\usepackage{amssymb}
\usepackage{amsfonts}

\usepackage{lineno}
\linenumbers

\usepackage[pdfencoding=auto, psdextra]{hyperref}

\usepackage{natbib}
\bibliographystyle{chicago}
\date{\today}

\usepackage{xspace}
\newcommand*{\ie}{i.e.\@\xspace}

\usepackage{color}

\newcommand{\Rx}[1]{\ensuremath{{\mathcal R}_{#1}}\xspace} 
\newcommand{\Ro}{\Rx{0}}
\newcommand{\Rc}{\Rx{\mathrm{c}}}
\newcommand{\Ri}{\Rx{\mathrm{i}}}
\newcommand{\RR}{\ensuremath{{\mathcal R}}\xspace}
\newcommand{\Rhat}{\ensuremath{{\hat\RR}}}
\newcommand{\Rnaive}{\ensuremath{{\mathcal R}_{\textrm{\tiny naive}}}\xspace}
\newcommand{\tsub}[2]{#1_{{\textrm{\tiny #2}}}}
\newcommand{\dd}[1]{\ensuremath{\, \mathrm{d}#1}}
\newcommand{\dtau}{\dd{\tau}}
\newcommand{\dx}{\dd{x}}
\newcommand{\dsigma}{\dd{\sigma}}

\newcommand{\tstart}{\ensuremath{\tsub{t}{start}}\xspace}
\newcommand{\tend}{\ensuremath{\tsub{t}{end}}\xspace}

\newcommand{\betaeff}{\ensuremath{\tsub{\beta}{eff}}\xspace}
\newcommand{\Keff}{\ensuremath{\tsub{K}{eff}}\xspace}

\newcommand{\pt}{p} %% primary time
\newcommand{\st}{s} %% secondary time

\newcommand{\psize}{{\mathcal P}} %% primary cohort size
\newcommand{\ssize}{{\mathcal S}} %% secondary cohort size

\newcommand{\gtime}{\sigma} %% generation interval
\newcommand{\gdist}{g} %% generation-interval distribution

\newcommand{\geff}{g_{\textrm{eff}}} %% generation-interval distribution

\newcommand{\total}{{\mathcal T}} %% total number of serial intervals

\newcommand{\PP}{{\mathcal P}}
\newcommand{\II}{{\mathcal I}}

\begin{document}

\begin{flushleft}{
	\Large
	\textbf\newline{
		Characterizing relative epidemic strength and speed of new SARS-CoV-2 variants
	}
}
\end{flushleft}

\section*{Abstract}

Characterizing epidemic dynamics of new SARS-CoV-2 variants is critical to predicting the future course of the pandemic.
Here, we explore how assumptions about the relative strength (i.e., ratio of reproduction numbers $\RR$) and relative speed (i.e., differences in growth rates $r$) of the new variant affect conclusions about each other, given the possibility that the new variant has longer generation intervals.
When relative speed---which corresponds to the growth rate of the logistic curve that models the proportion of the new variant---is held constant, higher growth rate of the original strain leads to higher relative strength.
When relative strength is held constant, however, higher reproduction number of the original strain leads to smaller relative speed.
Therefore, intervention strategies that reduce $\RR$ can cause the variant to replace the original strain faster.
Neglecting differences in generation intervals lead to opposite conclusions about the relationship between relative strength and speed, biasing their estimates.
We suggest that estimates of increased transmissibility of new variants should be re-assessed in the light of these findings.

\section{Introduction}

Since the emergence of the new SARS-CoV-2 variant of concern (VoC), estimating its epidemic strength and speed remains one of the key questions in controlling its spread \citep{switzerland2021variant, davies2021estimated, di2021impact, leung2021early, volz2021transmission,zhao2021}.
Epidemic strength, characterized by the reproduction number $\RR$, provides information about the final size of the epidemic as well as the amount of intervention required to eliminate the disease \citep{anderson1991infectious}.
Epidemic speed, characterized by the growth rate $r$ provides information about the population-level time scale of the epidemic and, therefore, when the new variant will replace the existing variant.
These two quantities are linked by generation intervals---defined as time between infection and transmission---which define individual-level time scale of the epidemic \citep{roberts2007model,svensson2007note,wallinga2007generation}.

Many studies have tried to estimate the strength of the novel variant from the observed speed \citep{davies2021estimated, leung2021early, volz2021transmission,zhao2021}.
On the other hand, some studies have assumed a value of the strength of the novel variant and tried to predict future spread (therefore, the speed) of the variant \citep{davies2021estimated}.
While both approaches are reasonable, they can lead to different conclusions.
In many contexts, previous studies have shown that inferences about disease dynamics (e.g., effectiveness of intervention strategy) can changes depending on whether $\RR$ or $r$ is held constant \citep{dushoff2020speed}.
Such inference also depends on assumptions about the underlying generation-interval distribution \citep{roberts2007model,svensson2007note,wallinga2007generation}.
While many studies assume that the generation-interval distributions do not differ between the existing strain and the variant, recent evidence suggests that the new variant has longer infectious period, leading to longer generation intervals.

Here, we explore how different assumptions about the new variant affects estimates of its strength and speed.
We ask how the estimates of relative strength (i.e., the ratio between the strength of the existing strain and the variant) and speed (i.e., the difference between the speed of the existing strain and the variant) depend on each other.
We then show how neglecting differences in the generation-interval distributions can lead to biased estimates.

\section{Renewal equation framework}

We use the renewal equation framework to characterize the spread of two pathogen strains---in this case, the wild type SARS-CoV-2 virus and the variant of concern.
Assuming that the effects mutations are negligible, the current incidence of infection $i_x(t)$ caused by the wild type ($x=w$) and the variant ($x=v$) can be expressed in terms of their previous incidence $i_x(t-\tau)$ and the rate at which secondary cases are generated at time $t$ by individuals infected $\tau$ time units ago $K_x(t, \tau)$:
\begin{equation}
i_x(t) = \int_0^\infty i_x(t-\tau) K_x(t, \tau) \dtau.
\end{equation}
This framework provides a flexible way of modeling disease dynamics and generalizes compartmental models, such as the SEIR model \citep{heesterbeek1996concept, diekmann2000mathematical, roberts2004modelling, aldis2005integral, roberts2007model, champredon2018equivalence}.

The integral of the kernel $\RR_x(t) = \int K_x(t, \tau) \dtau$---referred to as the instantaneous reproduction number---describes the average infectiousness of previously infected individuals at time $t$ \citep{fraser2007estimating}.
The normalized kernel $g_x(t, \tau) = K_x(t, \tau)/\RR_x(t)$---which we refer to as the instantaneous generation-interval distribution---describes their relative contribution to current incidence $i_x(t)$ and provides information about the time scale of disease transmission.
Both the reproduction number and the generation-interval distribution can depend on several factors, including intrinsic infectiousness of an infected individual, non-pharmaceutical interventions, awareness-driven behavior, and population-level susceptibility.
Many of these factors can affect between both the wild type and the variant in similar ways.

Over a short period of time, we can assume that epidemiological conditions remain roughly constant: $\RR_x(t) \approx \RR_x$ and $g_x(t, \tau) \approx g_x(\tau)$.
In this case, the incidence of both strains exhibit exponential growth or decay at rate $r_x$, satisfying the Euler-Lotka equation \citep{wallinga2007generation}:
\begin{equation}
\frac{1}{\RR_x} = \int_0^\infty \exp(- r_x \tau) g_x(\tau) \dtau.
\end{equation}
We can approximate the $r$--$\RR$ relationship by assuming that the generation-interval distribution is approximately gamma-distributed, and summarizing it using the mean generation interval $\bar{G}_x$ and the squared coefficient of variation $\kappa_x$ \citep{park2019practical}:
\begin{equation}
\RR_x \approx (1 + \kappa_x r_x \bar{G}_x)^{1/\kappa_x}.
\end{equation}
The gamma assumption covers models that assume exponentially distributed generation intervals (when $\kappa=1$), corresponding to the SIR model \citep{anderson1991infectious}; various gamma assumptions and approximations are widely used in epidemic modeling, including for models of SARS-CoV-2 \citep{doi:10.1098/rsif.2020.0144}.
We use this framework to investigate how inferences about strength and speed of the variant depend on our assumptions about the underlying generation-interval distributions.
For simplicity we neglect differences in the squared coefficient of variation and assume $\kappa_w = \kappa_v = \kappa$; instead, we focus on the effect of potential differences in the mean generation intervals.

\jd{I like to talk about `effective dispersion' when we're getting into the details of distributions and how good the gamma assumption is. I think at this level we're OK with saying we approximate by a gamma and kappa is the squared CV.}

\section{Inferring relative strength from relative speed}

Epidemic speed $r_x$ is often estimated directly from case data \citep{mills2004transmissibility,nishiura2009transmission,ma2014estimating}.
Studies of new SARS-CoV-2 variants, however, have mostly focused on characterizing changes in \emph{proportion} of a new variant \citep{switzerland2021variant, davies2021estimated, di2021impact, leung2021early, volz2021transmission,zhao2021}.
This has the advantage of correcting for testing patterns and intensity, and possibly for other transient effects that might affect variants and wildtype viruses similarly.
When incidence is changing exponentially, the proportion of the new variant $p(t)$ follows a logistic equation:
\begin{align}
p(t) &= \frac{i_v(0) \exp(r_v t)}{i_w(0) \exp(r_w t) + i_v(0) \exp(r_v t)},
\\ &= \frac{1}{1 + \left(i_w(0)/i_v(0)\right) \exp(-\delta t)},
\end{align}
where the logistic growth rate $\delta = r_v - r_w$ corresponds to the relative speed of the epidemic.

\begin{figure}[!th]
\includegraphics[width=\textwidth]{relstrength.pdf}
\caption{
\textbf{Relative strength of the new variant given observed relative speed.}
(A) Estimated relative strength of the new variant $\hat{\theta}$ given speed of the wild type
$r_w$ assuming identical generation-interval distributions.
(B) True relative strength of the new variant $\theta$ given speed of the wild type $r_w$ accounting for longer generation intervals of the new variant.
(C) Ratio between the true and the estimated relative strength of the new variant $\theta/\hat{\theta}$.
True relative strength of the new variant $\theta$ is calculated using $\delta=0.1\,\textrm{days}$, $\bar{G}_w = 5\,\textrm{days}$, and $\bar{G}_v = 7\,\textrm{days}$.
}
\label{fig:relstrength}
\end{figure}

As our baseline scenario, we assume $\delta = 0.1/\textrm{days}$ based on the observed relative speed of the new variant in the UK \citep{davies2021estimated}.
We then ask how the inference of relative strength $\theta = \RR_v/\RR_w$ of the new variant from its relative speed $\delta$ depends on assumptions about the generation-interval distribution.
The mean generation interval $\bar{G}_w$ of the wild type is assumed to equal 5 days \citep{ferretti2020quantifying}.
Previous estimates of squared coefficient of variation $\kappa$ in generation intervals are roughly to equal 0.2 \citep{ferretti2020quantifying}.
However, we expect there to be greater uncertainty in $\kappa$ than in $\bar{G}$, and so we evaluate the estimates of relative strength $\theta$ across a wide range of $\kappa$, ranging from 0 (delta distribution) to 1 (exponential distribution).

Previous studies have assumed that the generation-interval distributions do not differ between the wild type and the variant: $\bar{G}_v = \bar{G}_w = \bar{G}$.
In this case, the estimate of the relative strength is given by:
\begin{equation}
\hat{\theta} = \left(1 + \frac{\kappa \delta \bar{G}}{1 + \kappa r_w \bar{G}}\right)^{1/\kappa}.
\end{equation}
Therefore, if the wild type is spreading faster than we previously thought (higher $r_w$), we expect our estimate of the relative strength $\theta$ to go down (\fref{relstrength}A).
The relative strength also depends on the \emph{effective dispersion} in generation intervals (i.e., the amount of variability in generation intervals, as measured by $\kappa$)---using narrower distributions makes $\hat{\theta}$ less sensitive to $r_w$.

However, if the mean generation interval of the new variant is longer than that of the wild type ($\bar{G}_v > \bar{G}_w$), the true relative strength $\theta = \RR_v/\RR_w$ is given by:
\begin{equation}
\theta = \left(\frac{1 + \kappa (r_w + \delta) \bar{G}_v}{1 + \kappa r_w \bar{G}_w}\right)^{1/\kappa}.
\end{equation}
In this case, the effect of the underlying speed $r_w$ on the relative strength $\theta$ can change directions---\fref{relstrength}B illustrates this effect assuming $\bar{G}_v=7\,\textrm{days}$.
As $\kappa \to 0$, this approaches
\begin{equation}
\theta \to \exp(r_w (\bar{G}_v-\bar{G}_w)+\delta G_v), 
\end{equation}
meaning that $\theta$ increases with $r_w$ (\fref{relstrength}B).
On the other hand, as $\kappa \to 1$, we have
\begin{equation}
\theta \to \frac{1 + (r_w + \delta) \bar{G}_v}{1 + r_w \bar{G}_w} = \frac{\bar{G}_v}{\bar{G}_w} + \frac{1+ \delta \bar{G}_v  - \bar{G}_v/\bar{G}_w}{1 + r_w \bar{G}_w},
\end{equation}
which decreases with $r_w$ (\fref{relstrength}B).

Therefore, neglecting potential differences in generation intervals generally leads to under-estimation of the relative strength (\fref{relstrength}C).
In some cases, this can lead to over-estimation (see $r_w < -0.1$ in \fref{relstrength}C):
Longer generation intervals lead to slower exponential decay when $\RR < 1$, meaning that the new variant can still increase in proportion ($\delta > 0$) even when $\RR_v < \RR_w$.
However, this is unlikely to be the case as the new variant spread quickly under mixed conditions.
Instead, it seems more plausible that current estimates have under-estimated the true relative strength by 1.1---1.4 fold.

\section{Inferring relative speed from relative strength}

While relative speed is easier to measure, we often do not expect $\delta$ to be constant---this can depend on epidemic and demographic factors and can vary across time and space.
Instead, assuming a constant relative strength $\theta$ better matches biological hypotheses about the new variant (e.g., increased transmission and immune escape).
Several studies have tried to predict the future spread of the new variant, including when it will replace the wild type (therefore, its relative speed $\delta$).
As our baseline scenario we assume $\theta = 1.8$ and ask how assumptions about the generation-interval distribution affect predictions of $\delta$.


\begin{figure}[!th]
\includegraphics[width=\textwidth]{relspeed.pdf}
\caption{
\textbf{Relative speed of the new variant given assumed relative strength.}
(A) True relative speed of the new variant $\delta$ given strength of the wild type $\RR_w$ and the relative strength $\theta$.
The dashed contour line corresponds to $\delta = 0$.
True relative speed of the new variant $\delta$ is calculated using $\bar{G}_w = 5\,\textrm{days}$, $\bar{G}_v = 8\,\textrm{days}$, and $\kappa = 1/5$. 
(B) Estimated relative speed of the new variant $\hat{\delta}$ given strength of the wild type $\RR_w$ and the relative strength $\theta$ assuming identical generation-interval distributions.
(C) Biases in the estimates of the relative speed of the new variant $\hat{\delta} - \delta$ given strength of the wild type $\RR_w$ and the relative strength $\theta$.
The dashed contour line corresponds to $\hat{\delta} = \delta$.
}
\label{fig:relspeed}
\end{figure}

Like others, we first ignore possible differences between the generation-interval distributions of the wild type and the new variant and try to estimate $\delta$:
\begin{equation}
\hat{\delta} = \frac{(\theta \RR_w)^{\kappa_w} - \RR_w^{\kappa_w}}{\kappa_w \bar{G}_w}.
\end{equation}
In this case, the estimated $\hat{\delta}$ increases as the underlying strength of the wild type $\RR_w$ increases (\fref{relspeed}A).
Assuming a narrower distribution causes $\hat{\delta}$ to be less sensitive to $\RR_w$;
when $\kappa = 0$ (delta-distributed generation intervals), $\hat{\delta}$ remains constant.

When we allow the mean generation interval of the new variant to be long ($\bar{G}_v > \bar{G}_w$), the true $\theta$--$\delta$ relationship is given by:
\begin{equation}
\delta = \frac{(\theta \RR_w)^{\kappa} - 1}{\kappa \bar{G}_v} - \frac{\RR_w^{\kappa} - 1}{\kappa \bar{G}_w}.
\end{equation}
In this case, the direction of the effect of $\RR_w$ on $\delta$ can change depending on the value of $\kappa$.
As $\kappa \to 1$, we have
\begin{equation}
\delta \to \frac{(\bar{G}_w \theta - \bar{G}_v) \RR_w + \bar{G}_v - \bar{G}_w}{\bar{G}_v \bar{G}_w},
\end{equation}
which changes linearly with $\RR_w$ with a slope of $(\bar{G}_w \theta - \bar{G}_v)/\bar{G}_v \bar{G}_w$ (which can be either negative or positive).
For the values we consider, this increases with $\RR_w$ (\fref{relspeed}B).
On the other hand, as $\kappa \to 0$, we have
\begin{equation}
\delta \to \frac{\log(\theta \RR_w)}{\bar{G}_v} - \frac{\log(\RR_w)}{\bar{G}_w},
\end{equation}
which always decreases with $\RR_w$ provided that $\bar{G}_v > \bar{G}_w$ (\fref{relspeed}B).
Overall, estimates of $\delta$ become less sensitive to values of $\RR_w$ and range between 0.05/day and 0.15/day, which is consistent with the observed values of $\delta$ from data.
Therefore, assuming equal generation intervals generally leads to over-estimation of $\delta$ (\fref{relspeed}C).

\section{Discussion}

Here, we explored how the generation-interval distribution shapes the link between relative strength and speed of the new variant.
We define relative strength as the ratio between reproduction numbers ($\theta=\RR_v/\RR_w$) and relative speed as the difference between growth rates ($\delta=r_v-r_w$) of the variant and the wild type.
When the generation-interval distributions are asumed to be identical, we find that $\hat{\theta}$ decreases with $$

When relative speed $\delta$ is held constant, faster growth of the existing strain leads to larger relative strength.
When relative strength $\theta$ is held constant, higher strength of the existing strain leads to smaller relative speed.
Failing to account for the differences in the generation-interval distributions lead to opposite conclusions, leading to biased estimates of $\delta$ and $\theta$.

We argue that both perspectives (i.e., whether we hold relative strength or speed constant) are useful in understanding the dynamics of the new variant.
Early in the spread, we typically want to fix the relative speed because we can directly observe it.
As more information about the transmission and immunity profiles of the new variant becomes available, it is better to fix the relative strength as it better mathces biological assumptions.
We suggest researchers to be mindful about which quantity they hold constant and how their assumptions lead to their conclusions.

Our simple framework further reveals counterintuitive, but important, insight.
In general, one would expect that stronger intervention strategies is required in order to slow down the replacement of the original strain by the new variant.
However, this is only true if both have the same mean generation intervals (\fref{relspeed}B).
If the new variant has longer generation intervals than the original strain, imposing strong intervention (therefore reducing both $\RR_w$ and $\RR_v$ by a constant amount) results in a higher relative speed $\delta$, and therefore faster replacement of the original strain (\fref{relspeed}A).
While it is still necessary to impose stronger intervention to control the epidemic if the new variant has higher $\RR$, it may have unexpected consequences.

Since the beginning of the SARS-CoV-2 pandemic, the importance of understanding the role of generation-interval distributions in its spread has been constantly emphasized.
Yet, we still feel that generation intervals are under-appreciated, often viewed as a leeway to estimating $\RR$.
Future studies should consider the possibility that the new variant might have longer generation intervals and evaluate how assumptions about generation intervals affect their conclusions.

\bibliography{newvariant}

\end{document}
