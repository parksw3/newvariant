\documentclass[12pt]{article}
\usepackage[top=1in,left=1in, right = 1in, footskip=1in]{geometry}

\usepackage{graphicx}
\usepackage{xspace}
%\usepackage{adjustbox}

\newcommand{\comment}{\showcomment}
%% \newcommand{\comment}{\nocomment}

\newcommand{\showcomment}[3]{\textcolor{#1}{\textbf{[#2: }\textsl{#3}\textbf{]}}}
\newcommand{\nocomment}[3]{}

\newcommand{\jd}[1]{\comment{cyan}{JD}{#1}}
\newcommand{\swp}[1]{\comment{magenta}{SWP}{#1}}
\newcommand{\bmb}[1]{\comment{blue}{BMB}{#1}}
\newcommand{\djde}[1]{\comment{red}{DJDE}{#1}}

\newcommand{\eref}[1]{Eq.~\ref{eq:#1}}
\newcommand{\fref}[1]{Fig.~\ref{fig:#1}}
\newcommand{\Fref}[1]{Fig.~\ref{fig:#1}}
\newcommand{\sref}[1]{Sec.~\ref{#1}}
\newcommand{\frange}[2]{Fig.~\ref{fig:#1}--\ref{fig:#2}}
\newcommand{\tref}[1]{Table~\ref{tab:#1}}
\newcommand{\tlab}[1]{\label{tab:#1}}
\newcommand{\seminar}{SE\mbox{$^m$}I\mbox{$^n$}R}

\usepackage{amsthm}
\usepackage{amsmath}
\usepackage{amssymb}
\usepackage{amsfonts}

\usepackage{lineno}
\linenumbers

\usepackage[pdfencoding=auto, psdextra]{hyperref}

\usepackage{natbib}
\bibliographystyle{unsrt}
\date{\today}

\usepackage{xspace}
\newcommand*{\ie}{i.e.\@\xspace}

\usepackage{color}

\newcommand{\rx}[1]{\ensuremath{{r}_{#1}}\xspace} 
\newcommand{\ry}[1]{\rx{\mathrm{#1}}}

\newcommand{\Rx}[1]{\ensuremath{{\mathcal R}_{#1}}\xspace} 
\newcommand{\Ry}[1]{\Rx{\mathrm{#1}}}
\newcommand{\Ro}{\Rx{0}}
\newcommand{\Rc}{\Rx{\mathrm{c}}}
\newcommand{\Ri}{\Rx{\mathrm{i}}}
\newcommand{\RR}{\ensuremath{{\mathcal R}}\xspace}
\newcommand{\Rhat}{\ensuremath{{\hat\RR}}}
\newcommand{\Rnaive}{\ensuremath{{\mathcal R}_{\textrm{\tiny naive}}}\xspace}
\newcommand{\tsub}[2]{#1_{{\textrm{\tiny #2}}}}
\newcommand{\dd}[1]{\ensuremath{\, \mathrm{d}#1}}
\newcommand{\dtau}{\dd{\tau}}
\newcommand{\dx}{\dd{x}}
\newcommand{\dsigma}{\dd{\sigma}}

\newcommand{\tstart}{\ensuremath{\tsub{t}{start}}\xspace}
\newcommand{\tend}{\ensuremath{\tsub{t}{end}}\xspace}

\newcommand{\betaeff}{\ensuremath{\tsub{\beta}{eff}}\xspace}
\newcommand{\Keff}{\ensuremath{\tsub{K}{eff}}\xspace}

\newcommand{\pt}{p} %% primary time
\newcommand{\st}{s} %% secondary time

\newcommand{\psize}{{\mathcal P}} %% primary cohort size
\newcommand{\ssize}{{\mathcal S}} %% secondary cohort size

\newcommand{\gtime}{\sigma} %% generation interval
\newcommand{\gdist}{g} %% generation-interval distribution

\newcommand{\geff}{g_{\textrm{eff}}} %% generation-interval distribution

\newcommand{\total}{{\mathcal T}} %% total number of serial intervals

\newcommand{\PP}{{\mathcal P}}
\newcommand{\II}{{\mathcal I}}

\begin{document}

\begin{flushleft}{
	\Large
	\textbf\newline{
		Characterizing relative epidemic strength and speed of new SARS-CoV-2 variants
	}
}
\end{flushleft}

\section*{Abstract}

\section{Introduction}

Since the emergence of the new SARS-CoV-2 variant of concern (VoC), estimating its epidemic strength and speed remains one of the key questions in controlling its spread \citep{switzerland2021variant, davies2021estimated, di2021impact, leung2021early, volz2021transmission,zhao2021}.
Epidemic strength, characterized by the reproduction number $\RR$, provides information about the final size of the epidemic as well as the amount of intervention required to eliminate the disease \citep{anderson1991infectious}.
Epidemic speed, characterized by the growth rate $r$ provides information about the population-level time scale of the epidemic and, potentially provides information on when the new variant will replace the existing variant as the dominant strain.
Strength and speed are linked by generation intervals---defined as time between infection and transmission---which characterizes individual-level time scale of the epidemic \citep{roberts2007model,svensson2007note,wallinga2007generation}.

Many studies have tried to estimate the strength of the novel variant from the observed speed \citep{davies2021estimated, leung2021early, volz2021transmission,zhao2021}.
In contrast, some studies have assumed a value of the strength of the novel variant and tried to predict future spread (therefore, the speed) of the variant \citep{davies2021estimated}.
While both approaches are reasonable in principle, holding different quantities constant (i.e., strength or speed) can lead to different conclusions, in practice, about the spread of the disease, and therefore, its control.
The reason is that inference of speed from strength and strength from speed depends on assumptions about the generation-interval distribution \citep{roberts2007model,svensson2007note,wallinga2007generation,dushoff2020speed}.
For example, when we assume a fixed value of epidemic strength, longer generation intervals lead to faster epidemic (lower $r$), making the epidemic seem easier to control.
When we assume a fixed value of epidemic speed, longer generation intervals lead to stronger epidemic (higher $\RR$), making the epidemic seem more difficult to control.

In addition, most studies have assumed that the generation-interval distributions do not differ between the existing strain and the variant, but recent evidence suggests that the new variant may have a longer duration of infection: 13.3 days (90\% CI: 10.1--16.5) for the new variant and 8.2 days (90\% CI: 6.5--9.7) for the wild type \citep{kissler2021densely}.
This implies that the mean generation interval of the new variant is likely to be longer than that of the wild type.
Other studies have considered the possibility that the faster growth rate of the new variant may be driven, in part, by \emph{shorter} generation intervals \citep{davies2021estimated,volz2021transmission}.
If the generation-interval distribution of the new variant is different from the generation-interval distribution of the wild type, current estimates of the strength of the new variants may be biased.
More information is clearly needed, however. 
And the story may be complicated: generation intervals depend on many other biological and behavioral factors, including for example self-isolation after symptom onset leading to shorter generation intervals.

Here, we explore how different assumptions about the relative generation time of the new variant affect estimates of its strength $\RR$ and speed $r$.
We do this by exploring the relationship between relative strength (the ratio $\Ry{var}/\Ry{wt}$) and relative speed (the difference $\ry{var}-\ry{wt}$).
We find that neglecting differences in the generation-interval distributions can lead to biased estimates, which, in turn, influence mitigation priorities.
\jd{Just thoughts. I know we haven't introduced this terminology yet, and I know that you're going to use different terminology below. The point is that the old version was clunky and I'm trying to sharpen it. We can keep working.}

\section{Renewal equation framework}

We use the renewal equation framework to characterize the spread of two pathogen strains---in this case, the wild type SARS-CoV-2 virus and a focal variant of concern (e.g., B.1.1.7).
Neglecting the (relatively slow) rate of new mutations, the current incidence of infection $i_x(t)$ caused by the wild type (``wt'') and the variant (``var'') can be expressed in terms of their previous incidence $i_x(t-\tau)$ and the rate at which secondary cases are generated at time $t$ by individuals infected $\tau$ time units ago $K_x(t, \tau)$:
\begin{equation}
i_x(t) = \int_0^\infty i_x(t-\tau) K_x(t, \tau) \dtau.
\end{equation}
This framework provides a flexible way of modeling disease dynamics and generalizes compartmental models, such as the SEIR model \citep{heesterbeek1996concept, diekmann2000mathematical, roberts2004modelling, aldis2005integral, roberts2007model, champredon2018equivalence}.

The integral of the kernel $\RR_x(t) = \int K_x(t, \tau) \dtau$---referred to as the instantaneous reproduction number---describes the average infectiousness of previously infected individuals at time $t$ \citep{fraser2007estimating}.
\jd{This is a particular kind of weighted average – in particular, it's not weighted by the actual number of infected individuals present. I kind of want to say GI-weighted average, but then we would have to introduce GI as an abbrv.}
The normalized kernel $g_x(t, \tau) = K_x(t, \tau)/\RR_x(t)$---which we refer to as the instantaneous generation-interval distribution---describes their relative contribution to current incidence $i_x(t)$ and provides information about the time scale of disease transmission.
Both the reproduction number and the generation-interval distribution can depend on several factors, including intrinsic infectiousness of an infected individual, non-pharmaceutical interventions, awareness-driven behavior, and population-level susceptibility \citep{fraser2007estimating}.

Over a short period of time, we can assume that epidemiological conditions remain roughly constant: $\RR_x(t) \approx \RR_x$ and $g_x(t, \tau) \approx g_x(\tau)$.
In this case, the incidence of each strain will exhibit exponential growth (or decay) at rate $r_x$, satisfying the Euler-Lotka equation \citep{wallinga2007generation}:
\begin{equation}
\frac{1}{\RR_x} = \int_0^\infty \exp(- r_x \tau) g_x(\tau) \dtau.
\end{equation}
We can approximate the $r$--$\RR$ relationship by assuming that the generation-interval distribution is approximately gamma-distributed, and summarizing it using the mean generation interval $\bar{G}_x$ and the squared coefficient of variation $\kappa_x$ \citep{park2019practical}:
\begin{equation}
\RR_x \approx (1 + \kappa_x r_x \bar{G}_x)^{1/\kappa_x}.
\end{equation}
The gamma assumption covers models that assume exponentially distributed generation intervals (when $\kappa=1$), corresponding to the SIR model \citep{anderson1991infectious}; various gamma assumptions and approximations are widely used in epidemic modeling, including for models of SARS-CoV-2 \citep{doi:10.1098/rsif.2020.0144}.
We use this framework to investigate how inferences about strength and speed of the variant depend on our assumptions about the underlying generation-interval distributions.
For simplicity we neglect differences in the squared coefficient of variation and assume $\kappa_w = \kappa_v = \kappa$; instead, we focus on the effect of potential differences in the mean generation intervals.

\section{Inferring relative strength from relative speed}

Epidemic speed $r_x$ is often estimated directly from case data \citep{mills2004transmissibility,nishiura2009transmission,ma2014estimating}.
Studies of new SARS-CoV-2 variants, however, have mostly focused on characterizing changes in \emph{proportion} of a new variant \citep{switzerland2021variant, davies2021estimated, di2021impact, leung2021early, volz2021transmission,zhao2021}.
\jd{I probably added patterns below, but now I want to drop it. Sometimes the testing pattern is: we just noticed the new variant and so let's go and test intensively in the communities where we think it is.}
This has the advantage of correcting for testing patterns and intensity, and possibly for other transient effects that might affect variants and wild type viruses similarly.
When incidence is changing exponentially ($i_x(t) = i_x(t_0) \exp(r_x t)$), the proportion of the new variant $p(t)$ follows a logistic equation:
\begin{align}
p(t) &= \frac{i_v(t_0) \exp(r_v t)}{i_w(t_0) \exp(r_w t) + i_v(t_0) \exp(r_v t)},
\\ &= \frac{1}{1 + \left(i_w(t_0)/i_v(t_0)\right) \exp(-\delta t)},
\end{align}
where the logistic growth rate $\delta = r_v - r_w$ corresponds to the relative speed of the epidemic.

\begin{figure}[!th]
\includegraphics[width=\textwidth]{relstrength.pdf}
\caption{
\textbf{Relative strength of the new variant given observed relative speed.}
(A) Estimated relative strength of the new variant $\hat{\theta}$ given speed of the wild type
$r_w$ assuming identical generation-interval distributions.
(B) True relative strength of the new variant $\theta$ given speed of the wild type $r_w$ and ratio between the mean generation interval of the new variant $\bar{G}_v$ and that of the wild type $\bar{G}_w$.
(C) Ratio between the true and the estimated relative strength of the new variant $\theta/\hat{\theta}$.
True relative strength of the new variant $\theta$ is calculated using $\delta=0.1\,\textrm{days}$, $\bar{G}_w = 5\,\textrm{days}$, and $\kappa = 1/5$.
}
\label{fig:relstrength}
\end{figure}

As our baseline scenario, we assume $\delta = 0.1/\textrm{days}$ based on the observed relative speed of the new variant in the UK \citep{davies2021estimated}.
This speed can be then used to estimate the relative strength $\theta = \RR_v/\RR_w$ of the new variant with a caveat; such inference depends on assumptions about the generation-interval distribution.
The mean generation interval $\bar{G}_w$ of the wild type is assumed to equal 5 days \citep{ferretti2020quantifying}.
The squared coefficient of variation $\kappa$ in generation intervals os assumed to equal 0.2 \citep{ferretti2020quantifying}---for simplicity, we assume that both the variant and the wild type have equal $\kappa$ and only consider differences in the mean.
As we expect there to be greater uncertainty in $\kappa$ than in $\bar{G}$, we evaluate the estimates of relative strength $\theta$ across a wide range of $\kappa$, ranging from 0 (delta distribution) to 1 (exponential distribution).

Previous studies have assumed that the generation-interval distributions do not differ between the wild type and the variant: $\bar{G}_v = \bar{G}_w = \bar{G}$.
In this case, the estimate of the relative strength is given by \citep{park2019practical}:
\begin{equation}
\hat{\theta} = \left(1 + \frac{\kappa \delta \bar{G}}{1 + \kappa r_w \bar{G}}\right)^{1/\kappa}.
\end{equation}
Therefore, estimates of relative strength $\theta$ depends not only on the relative speed $\delta$ but also on how fast the wild type is spreading in the population $r_w$---a few analyses have implicitly or explicitly neglected this factor by either assuming $r_w = 0$ \citep{switzerland2021variant} or $\kappa = 0$ \citep{davies2021estimated}, in which case we get $\hat{\theta} = \exp(\delta \bar{G})$.
Instead, if the wild type was spreading faster than previously assumed (higher $r_w$), the estimates of the relative strength $\theta$ will be go down (\fref{relstrength}A).
The relative strength also depends on the \emph{effective dispersion} in generation intervals (i.e., the amount of variability in generation intervals, as measured by $\kappa$)---using narrower distributions makes $\hat{\theta}$ less sensitive to $r_w$.

However, if the true mean generation interval of the new variant is different from that of the wild type ($\bar{G}_v \neq \bar{G}_w$), the true relative strength $\theta = \RR_v/\RR_w$ is given by:
\begin{equation}
\theta = \left(\frac{1 + \kappa (r_w + \delta) \bar{G}_v}{1 + \kappa r_w \bar{G}_w}\right)^{1/\kappa}.
\end{equation}
In this case, longer (or shorter) mean generation interval of the new variant generally translates to higher (or lower) values of $\theta$---\fref{relstrength}B illustrates this effect assuming $\kappa = 1/5$.
However, this direction can change when both the new variant and the wild type are decaying exponentially (see area below $r_v = 0$ line in \fref{relstrength}B)---when disease incidence is decaying exponentially, longer generation intervals lead to a \emph{lower} value of strength (and therefore lower relative strength $\theta$).

Therefore, neglecting potential differences in generation intervals can lead to both under- and over-estimation of the relative strength.
In \fref{relstrength}C, we quantify the degrees of bias using the ratio between the true value $\theta$, which accounts for differences in generation-interval distributions, and the estimated value $\hat{\theta}$, which assumes equal generation-interval distributions.
Across the ranges of parameters that we consider, we find that up to 1.4 fold (either upward or downward) adjustments to current estimates may be required.

\section{Inferring relative speed from relative strength}

While relative speed is easier to measure, we often do not expect $\delta$ to be constant---this can depend on epidemic and demographic factors and can vary across time and space.
Instead, biological mechanisms that cause new variant to have higher strength better translate to assuming a constant value of relative strength $\theta$.
For example, if the new variant has a higher transmission rate, the ratio between the strength of the new variant and that of the wild type will remain constant ($=\theta$) under strength-like interventions that reduces the strength of both strains by a constant amount.
However, this is not necessarily true under speed-like interventions (e.g., contact tracing and self-isolation), which depends on time since infection: if the generation-interval distributions are different, the effectiveness of interventions will also differ.
For now, we consider consequences of assuming a constant relative strength $\theta$ and predicting the relative speed $\delta$---an assumption often made (either implicitly or explicitly) in studies that have tried to predict the future spread of the new variant, including when it will replace the wild type.
As our baseline scenario we assume $\theta = 1.5$.

\begin{figure}[!th]
\includegraphics[width=\textwidth]{relspeed.pdf}
\caption{
\textbf{Relative speed of the new variant given assumed relative strength.}
(A) Estimated relative speed of the new variant $\hat{\delta}$ given strength of the wild type $\RR_w$ assuming identical generation-interval distributions.
(B) True relative strength of the new variant $\delta$ given strength of the wild type $\RR_w$ and ratio between the mean generation interval of the new variant $\bar{G}_v$ and that of the wild type $\bar{G}_w$..
(C) Difference between the true and the estimated relative speed of the new variant $\delta - \hat{\delta}$.
True relative speed of the new variant $\delta$ is calculated using $\theta=1.5$, $\bar{G}_w = 5\,\textrm{days}$, and $\kappa=1/5$.
}
\label{fig:relspeed}
\end{figure}

Like others, we first ignore possible differences between the generation-interval distributions of the wild type and the new variant and try to estimate $\delta$:
\begin{equation}
\hat{\delta} = \frac{(\theta \RR_w)^{\kappa} - \RR_w^{\kappa}}{\kappa \bar{G}_w}.
\end{equation}
Here, we see that the estimated $\hat{\delta}$ increases as the underlying strength of the wild type $\RR_w$ increases (\fref{relspeed}A).
In other words, we expect the relative speed of the new variant to change as epidemiological conditions change, suggesting that a logistic model that assumes constant growth rate may not be appropriate for modeling the proportion of the new variant.
A narrower distribution causes $\hat{\delta}$ to be less sensitive to $\RR_w$;
when $\kappa = 0$ (delta-distributed generation intervals), $\hat{\delta}$ remains constant.

When we allow the mean generation interval of the new variant to be different ($\bar{G}_v \neq \bar{G}_w$), the true $\theta$--$\delta$ relationship under gamma approximation is given by \citep{park2019practical}:
\begin{equation}
\delta = \frac{(\theta \RR_w)^{\kappa} - 1}{\kappa \bar{G}_v} - \frac{\RR_w^{\kappa} - 1}{\kappa \bar{G}_w}.
\end{equation}
In this case, $\delta$ generally decreases as the mean generation interval of the new variant increases, except when both $\RR_v < 1$ and $\RR_w < 1$ (\fref{relspeed}B).
In some extreme cases (e.g., when $\bar{G}_v$ and $\RR_v$ are very low or when $\bar{G}_v$ and $\RR_v$ are very large), we can obtain $\delta < 0$---an unlikely scenario given that fast replacement of the new variant have been observed under mixed conditions.
This suggests that differences in generation intervals between the new variant and the wild type are likely to be not too big.
Finally, assuming equal generation intervals generally leads to both under- and over-estimation of $\delta$ (\fref{relspeed}C).

\section{Inferring relative strength from incidence data, in theory}

In practice, epidemiological conditions change over time due to intervention measures and susceptible depletion effects.
Such changes can be characterized by ``time-varying'' or ``instantaneous'' reproduction numbers, $\RR(t)$ \citep{fraser2007estimating}.
Assuming that the generation-interval distribution does not change over time, the instantaneous reproduction numbers of the new variant and of the wild type can be estimated from their corresponding incidence curves---an approach popularized by \cite{cori2013new}:
\begin{equation}
\RR_x(t) = \frac{i_x(t)}{\int_0^\infty i_x(t-\tau) g_x(\tau) \dtau}.
\end{equation}
Under strength-like intervention measures that reduce transmission rates by a constant amount, we expect ratios between true reproduction numbers to remain constant and correspond to the true relative strength: $\RR_v(t)/\RR_w(t) = \theta$.
However, if the assumed generation-interval distribution $\hat{g}(\tau)$ is different from the true distribution, then the ratio between the estimated reproduction numbers $\hat{\theta}(t) = \hat{\RR}_v(t)/\hat{\RR}_w(t)$ need not be constant, where
\begin{equation}
\hat{\RR}_x(t) = \frac{i_x(t)}{\int_0^\infty i_x(t-\tau) \hat{g}(\tau) \dtau}.
\end{equation}

Here, we simulate three different scenarios using a two strain renewal equation that assumes perfect cross-immunity (\fref{Rtbias}): 
(1) the wild type and the variant have the same generation-interval distributions, which match the assumed distribution (\fref{Rtbias}A--D);
(2) the assumed distribution matches the generation-interval distribution of the wild type but has a longer mean than the mean generation interval of the variant (\fref{Rtbias}E--H); and
(3) the assumed distribution matches the generation-interval distribution of the wild type but has a shorter mean than the mean generation interval of the variant (\fref{Rtbias}I--L).
Then, we compare the estimated ratio $\hat{\theta}(t) = \hat{\RR}_v(t)/\hat{\RR}_w(t)$ with the true ratio $\theta = \RR_v(t)/\RR_w(t)$.
For simplicity, we did not model non-pharmaceutical interventions;
instead, we let epidemic dynamics to be determined by susceptible depletion alone.

\begin{figure}[!th]
\includegraphics[width=\textwidth]{Rtbias.pdf}
\caption{
\textbf{Estimates of relative strength over time under different scenarios.}
(A, E, I) Daily incidence of infection caused by the new variant and the wild type.
(B, F, J) True (solid lines) and estimated (dashed lines) reproduction numbers of the new variant and the wild type over time.
(C, G, K) True (purple) and estimated (orange) ratios between reproduction numbers of the new variant and the wild type over time.
(D, H, L) Relationship between estimated reproduction numbers on a log-log sacle.
The dashed lines represent the one-to-one line. 
The dotted lines represent the $\RR_v=\theta \RR_w$ line. 
Top row: the assumed mean generation-interval (5 days) is equal to the mean generation-interval of the wild type and the variant.
Middle row: the assumed mean generation-interval (5 days) is equal to the mean generation-interval of the wild type but is longer than that of the varant (3.5 days).
Bottom row: the assumed mean generation-interval (5 days) is equal to the mean generation-interval of the wild type but is longer than that of the varant (6.5 days).
For all simulations, squared coefficient of variation in generation intervals is assumed to equal $\kappa = 1/5$.
}
\label{fig:Rtbias}
\end{figure}

When the assumed distribution matches the true distributions, the estimated ratio remains constant (\fref{Rtbias}C).
However, if the generation-interval distribution of the variant is different from the assumed distribution, the ratio changes over time (\fref{Rtbias}G,K).
In particular, if the true generation intervals have a shorter (or longer) mean than the assumed distribution, we over-estimate (or under-estimate) $\RR_v(t)$ during the growth phase and under-estimate (or over-estimate) $\RR_v(t)$ during the decay phase (\fref{Rtbias}F,J), which further translates to biases in the estimated relative strength (\fref{Rtbias}G,K).

We can also assess these biases by plotting the estimated strength of the variant $\hat{\RR}_v(t)$ against the estimated strength of the wild type $\hat{\RR}_w(t)$ on a log-log scale (\fref{Rtbias}D,H,L).
When generation-interval distributions are correctly specified, we obtain a straight line parallel to the one-to-one line. 
In this case, the slope corresponds to the logarithm of the relative strength $\log(\theta)$ (\fref{Rtbias}D).
When the assumed mean generation interval is longer (or shorter) than the that of the variant, we over-estimate (or under-estimate) the slope (\fref{Rtbias}H,L) on a log-log plot.

\section{Inferring relative strength from case data, in practice}

Seb Funk's analysis here.

\section{Discussion}

Here, we explored how the generation-interval distribution shapes the link between relative strength and speed of the new variant.
We define relative strength as the ratio between reproduction numbers ($\theta=\RR_v/\RR_w$) and relative speed as the difference between growth rates ($\delta=r_v-r_w$) of the variant and the wild type.
When the generation-interval distributions are assumed to be identical, we find that estimates of $\hat{\theta}$ decreases with $r_w$ whereas $\hat{\delta}$ increase with $\RR_w$.
When we account for possible differences in generation-interval distributions, these relationships can change directions and further lead to biases in these estimates.

We argue that both perspectives (i.e., whether we hold relative strength or speed constant) are useful in understanding the dynamics of the new variant.
Early in the spread, we typically want to fix the relative speed because we can directly observe it.
As more information about the transmission and immunity profiles of the new variant becomes available, it is better to fix the relative strength as it better matches biological mechanisms of its spread.
We suggest researchers to be mindful about which quantity they hold constant and how their assumptions lead to their conclusions.

Our simple framework further reveals counterintuitive, but important, insights on how to control the spread of both wild type and variants.
In general, one would expect that stronger intervention strategies are required in order to slow down the replacement of the original strain by the new variant.
However, this is only true if both have the same mean generation intervals (\fref{relspeed}B).
If the new variant has longer generation intervals than the original strain, imposing strong intervention (therefore reducing both $\RR_w$ and $\RR_v$ by a constant amount) can result in a higher relative speed $\delta$, and therefore faster replacement of the original strain (\fref{relspeed}B).
This is not necessarily a bad outcome as the absolute speed ($r_v$) of the new variant still decreases as $\RR_v$ decreases.
These effects should be taken into consideration for accurate projection of the spread of the new variant.

Even though SARS-CoV-2 has been spreading for more than a year, there is still considerable uncertainty in its generation intervals.
Observational studies typically focus on serial intervals (i.e., time between symptom onset of the infector and the infectee; \cite{svensson2007note}) because they are easier to measure \citep{griffin2020rapid}, but they are subject to dynamical biases that can be difficult to tease apart \citep{park2021forward}.
A few studies have tried to estimate the generation-interval distribution from serial intervals with means ranging between 3--6 days and squared coefficient of variations ranging between 0.1--0.6 \citep{ferretti2020quantifying,Ferretti2020timing,ganyani2020estimating,knight2020estimating}; 
however, most estimates neglect dynamical biases in serial intervals.
In addition, serial intervals do not account for asymptomatic transmission, adding further uncertainty to inferences of speed and strength \citep{park2020time}.

Since the beginning of the SARS-CoV-2 pandemic, a few studies have emphasized  the relevance of leveraging generation interval distributions to improve estimates of strength (e.g., \cite{doi:10.1098/rsif.2020.0144}) and the relative importance of modes of transmission (e.g., asymptomatic vs. symptomatic, \cite{park2020time}).
These same lessons have relevance for assessing the spread of variants in a partially susceptible population and improving efforts to control spread.
Future studies should prioritize detailded assessment of the relative generation intervals of SARS-CoV-2 and widespread variants, as well as evaluate how uncertainty in assumptions about generational intervals might bias conclusions.

\bibliography{newvariant}

\end{document}
