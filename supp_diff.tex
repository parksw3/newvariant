\documentclass[12pt]{article}
%DIF LATEXDIFF DIFFERENCE FILE


\usepackage[top=1in,left=1in, right = 1in, footskip=1in]{geometry}

\usepackage{graphicx}
\usepackage{xspace}
%\usepackage{adjustbox}

\newcommand{\comment}{\showcomment}
%% \newcommand{\comment}{\nocomment}

\newcommand{\showcomment}[3]{\textcolor{#1}{\textbf{[#2: }\textsl{#3}\textbf{]}}}
\newcommand{\nocomment}[3]{}

\newcommand{\jd}[1]{\comment{cyan}{JD}{#1}}
\newcommand{\swp}[1]{\comment{magenta}{SWP}{#1}}
\newcommand{\bmb}[1]{\comment{blue}{BMB}{#1}}
\newcommand{\djde}[1]{\comment{red}{DJDE}{#1}}

\newcommand{\eref}[1]{Eq.~\ref{eq:#1}}
\newcommand{\fref}[1]{Fig.~\ref{fig:#1}}
\newcommand{\Fref}[1]{Fig.~\ref{fig:#1}}
\newcommand{\sref}[1]{Sec.~\ref{#1}}
\newcommand{\frange}[2]{Fig.~\ref{fig:#1}--\ref{fig:#2}}
\newcommand{\tref}[1]{Table~\ref{tab:#1}}
\newcommand{\tlab}[1]{\label{tab:#1}}
\newcommand{\seminar}{SE\mbox{$^m$}I\mbox{$^n$}R}

\usepackage{amsthm}
\usepackage{amsmath}
\usepackage{amssymb}
\usepackage{amsfonts}

\usepackage{lineno}
\linenumbers

\usepackage[pdfencoding=auto, psdextra]{hyperref}

\usepackage{natbib}
\setcitestyle{numbers} 
\setcitestyle{square}
\bibliographystyle{prsb}
\date{\today}

\usepackage{xspace}
\newcommand*{\ie}{i.e.\@\xspace}

\usepackage{color}

%% Consistent, changeable style for subscripts
\newcommand{\vvvar}{\mathrm{var}}
\newcommand{\wwwt}{\mathrm{wt}}

\newcommand{\rx}[1]{\ensuremath{{r}_{#1}}\xspace} 
\newcommand{\ry}[1]{\rx{\mathrm{#1}}} 
\newcommand{\rw}{\rx{\wwwt}}
\newcommand{\rv}{\rx{\vvvar}}

\newcommand{\Rx}[1]{\ensuremath{{\mathcal R}_{#1}}\xspace} 
\newcommand{\Ry}[1]{\Rx{\mathrm{#1}}}
\newcommand{\Ro}{\Rx{0}}
\newcommand{\RR}{\ensuremath{{\mathcal R}}\xspace}
\newcommand{\Rw}{\Rx{\wwwt}}
\newcommand{\Rv}{\Rx{\vvvar}}
%DIF 64a64-65
\newcommand{\hRw}{\hat{\RR}_{\wwwt}} %DIF > 
\newcommand{\hRv}{\hat{\RR}_{\vvvar}} %DIF > 
%DIF -------

\newcommand{\days}{\ensuremath{\, \textrm{days}}}
\newcommand{\pday}{\ensuremath{/\textrm{day}}}
\newcommand{\dd}[1]{\ensuremath{\, \mathrm{d}#1}}
\newcommand{\dtau}{\dd{\tau}}
\newcommand{\dx}{\dd{x}}
\newcommand{\dsigma}{\dd{\sigma}}

\newcommand{\ix}[1]{\ensuremath{{i}_{#1}}\xspace} 
\newcommand{\iy}[1]{\ix{\mathrm{#1}}}
\newcommand{\iw}{\ix{\wwwt}}
\newcommand{\iv}{\ix{\vvvar}}

\newcommand{\Gx}[1]{\ensuremath{{\bar G}_{#1}}\xspace} 
\newcommand{\Gy}[1]{\Gx{\mathrm{#1}}}
\newcommand{\Gw}{\Gx{\wwwt}}
\newcommand{\Gv}{\Gx{\vvvar}}

\newcommand{\tsub}[2]{#1_{{\textrm{\tiny #2}}}}
\newcommand{\tstart}{\ensuremath{\tsub{t}{start}}\xspace}
\newcommand{\tend}{\ensuremath{\tsub{t}{end}}\xspace}

\newcommand{\betaeff}{\ensuremath{\tsub{\beta}{eff}}\xspace}
\newcommand{\Keff}{\ensuremath{\tsub{K}{eff}}\xspace}

\newcommand{\pt}{p} %% primary time
\newcommand{\st}{s} %% secondary time

\newcommand{\psize}{{\mathcal P}} %% primary cohort size
\newcommand{\ssize}{{\mathcal S}} %% secondary cohort size

\newcommand{\gtime}{\sigma} %% generation interval
\newcommand{\gdist}{g} %% generation-interval distribution

\newcommand{\geff}{g_{\textrm{eff}}} %% generation-interval distribution

\newcommand{\total}{{\mathcal T}} %% total number of serial intervals

\newcommand{\PP}{{\mathcal P}}
\newcommand{\II}{{\mathcal I}}
%DIF PREAMBLE EXTENSION ADDED BY LATEXDIFF
%DIF UNDERLINE PREAMBLE %DIF PREAMBLE
\RequirePackage[normalem]{ulem} %DIF PREAMBLE
\RequirePackage{color}\definecolor{RED}{rgb}{1,0,0}\definecolor{BLUE}{rgb}{0,0,1} %DIF PREAMBLE
\providecommand{\DIFaddtex}[1]{{\protect\color{blue}\uwave{#1}}} %DIF PREAMBLE
\providecommand{\DIFdeltex}[1]{{\protect\color{red}\sout{#1}}}                      %DIF PREAMBLE
%DIF SAFE PREAMBLE %DIF PREAMBLE
\providecommand{\DIFaddbegin}{} %DIF PREAMBLE
\providecommand{\DIFaddend}{} %DIF PREAMBLE
\providecommand{\DIFdelbegin}{} %DIF PREAMBLE
\providecommand{\DIFdelend}{} %DIF PREAMBLE
%DIF FLOATSAFE PREAMBLE %DIF PREAMBLE
\providecommand{\DIFaddFL}[1]{\DIFadd{#1}} %DIF PREAMBLE
\providecommand{\DIFdelFL}[1]{\DIFdel{#1}} %DIF PREAMBLE
\providecommand{\DIFaddbeginFL}{} %DIF PREAMBLE
\providecommand{\DIFaddendFL}{} %DIF PREAMBLE
\providecommand{\DIFdelbeginFL}{} %DIF PREAMBLE
\providecommand{\DIFdelendFL}{} %DIF PREAMBLE
%DIF HYPERREF PREAMBLE %DIF PREAMBLE
\providecommand{\DIFadd}[1]{\texorpdfstring{\DIFaddtex{#1}}{#1}} %DIF PREAMBLE
\providecommand{\DIFdel}[1]{\texorpdfstring{\DIFdeltex{#1}}{}} %DIF PREAMBLE
\newcommand{\DIFscaledelfig}{0.5}
%DIF HIGHLIGHTGRAPHICS PREAMBLE %DIF PREAMBLE
\RequirePackage{settobox} %DIF PREAMBLE
\RequirePackage{letltxmacro} %DIF PREAMBLE
\newsavebox{\DIFdelgraphicsbox} %DIF PREAMBLE
\newlength{\DIFdelgraphicswidth} %DIF PREAMBLE
\newlength{\DIFdelgraphicsheight} %DIF PREAMBLE
% store original definition of \includegraphics %DIF PREAMBLE
\LetLtxMacro{\DIFOincludegraphics}{\includegraphics} %DIF PREAMBLE
\newcommand{\DIFaddincludegraphics}[2][]{{\color{blue}\fbox{\DIFOincludegraphics[#1]{#2}}}} %DIF PREAMBLE
\newcommand{\DIFdelincludegraphics}[2][]{% %DIF PREAMBLE
\sbox{\DIFdelgraphicsbox}{\DIFOincludegraphics[#1]{#2}}% %DIF PREAMBLE
\settoboxwidth{\DIFdelgraphicswidth}{\DIFdelgraphicsbox} %DIF PREAMBLE
\settoboxtotalheight{\DIFdelgraphicsheight}{\DIFdelgraphicsbox} %DIF PREAMBLE
\scalebox{\DIFscaledelfig}{% %DIF PREAMBLE
\parbox[b]{\DIFdelgraphicswidth}{\usebox{\DIFdelgraphicsbox}\\[-\baselineskip] \rule{\DIFdelgraphicswidth}{0em}}\llap{\resizebox{\DIFdelgraphicswidth}{\DIFdelgraphicsheight}{% %DIF PREAMBLE
\setlength{\unitlength}{\DIFdelgraphicswidth}% %DIF PREAMBLE
\begin{picture}(1,1)% %DIF PREAMBLE
\thicklines\linethickness{2pt} %DIF PREAMBLE
{\color[rgb]{1,0,0}\put(0,0){\framebox(1,1){}}}% %DIF PREAMBLE
{\color[rgb]{1,0,0}\put(0,0){\line( 1,1){1}}}% %DIF PREAMBLE
{\color[rgb]{1,0,0}\put(0,1){\line(1,-1){1}}}% %DIF PREAMBLE
\end{picture}% %DIF PREAMBLE
}\hspace*{3pt}}} %DIF PREAMBLE
} %DIF PREAMBLE
\LetLtxMacro{\DIFOaddbegin}{\DIFaddbegin} %DIF PREAMBLE
\LetLtxMacro{\DIFOaddend}{\DIFaddend} %DIF PREAMBLE
\LetLtxMacro{\DIFOdelbegin}{\DIFdelbegin} %DIF PREAMBLE
\LetLtxMacro{\DIFOdelend}{\DIFdelend} %DIF PREAMBLE
\DeclareRobustCommand{\DIFaddbegin}{\DIFOaddbegin \let\includegraphics\DIFaddincludegraphics} %DIF PREAMBLE
\DeclareRobustCommand{\DIFaddend}{\DIFOaddend \let\includegraphics\DIFOincludegraphics} %DIF PREAMBLE
\DeclareRobustCommand{\DIFdelbegin}{\DIFOdelbegin \let\includegraphics\DIFdelincludegraphics} %DIF PREAMBLE
\DeclareRobustCommand{\DIFdelend}{\DIFOaddend \let\includegraphics\DIFOincludegraphics} %DIF PREAMBLE
\LetLtxMacro{\DIFOaddbeginFL}{\DIFaddbeginFL} %DIF PREAMBLE
\LetLtxMacro{\DIFOaddendFL}{\DIFaddendFL} %DIF PREAMBLE
\LetLtxMacro{\DIFOdelbeginFL}{\DIFdelbeginFL} %DIF PREAMBLE
\LetLtxMacro{\DIFOdelendFL}{\DIFdelendFL} %DIF PREAMBLE
\DeclareRobustCommand{\DIFaddbeginFL}{\DIFOaddbeginFL \let\includegraphics\DIFaddincludegraphics} %DIF PREAMBLE
\DeclareRobustCommand{\DIFaddendFL}{\DIFOaddendFL \let\includegraphics\DIFOincludegraphics} %DIF PREAMBLE
\DeclareRobustCommand{\DIFdelbeginFL}{\DIFOdelbeginFL \let\includegraphics\DIFdelincludegraphics} %DIF PREAMBLE
\DeclareRobustCommand{\DIFdelendFL}{\DIFOaddendFL \let\includegraphics\DIFOincludegraphics} %DIF PREAMBLE
%DIF END PREAMBLE EXTENSION ADDED BY LATEXDIFF

\begin{document}

\begin{flushleft}{
	\Large
	\textbf\newline{
		Roles of generation-interval distributions in shaping relative epidemic strength, speed, and control of new SARS-CoV-2 variants---Supplementary Materials
	}
}
\newline
\\
Sang Woo Park\textsuperscript{1,*}
Benjamin M.\ Bolker\textsuperscript{2,3,4}
Sebastian Funk\textsuperscript{5,6}
C.\ Jessica E.\ Metcalf\textsuperscript{1,7}
Joshua S.\ Weitz\textsuperscript{8,9}
Bryan T.\ Grenfell\textsuperscript{1,7,10}
Jonathan Dushoff\textsuperscript{2,3,4}
\\
\bigskip
\textbf{1} Department of Ecology and Evolutionary Biology, Princeton University, Princeton, NJ, USA
\\
\textbf{2} Department of Biology, McMaster University, Hamilton, ON, Canada
\\
\textbf{3} Department of Mathematics and Statistics, McMaster University, Hamilton, ON, Canada
\\
\textbf{4} M.\,G.\,DeGroote Institute for Infectious Disease Research, McMaster University, Hamilton, ON, Canada
\\
\textbf{5} Department for Infectious Disease Epidemiology, London School of Hygiene and Tropical Medicine, London, UK
\\
\textbf{6} Centre for Mathematical Modelling of Infectious Diseases, London School of Hygiene and Tropical Medicine, London, UK
\\
\textbf{7} Princeton School of Public and International Affairs, Princeton University, Princeton, NJ, USA
\\
\textbf{8} School of Biological Sciences, Georgia Institute of Technology, Atlanta, GA, USA
\\
\textbf{9} School of Physics, Georgia Institute of Technology, Atlanta, GA, USA
\\
\bigskip

*Corresponding author: swp2@princeton.edu
\bigskip

\end{flushleft}

%% 6635 words

\section{Supplementary Text}

\subsection{Two-strain renewal equation}

We use a two-strain renewal equation to simulate the spread of the variant and the wild type \DIFdelbegin \DIFdel{.
}\DIFdelend \DIFaddbegin \DIFadd{in Sections 5--6.
Throughout this section, we assume that the intrinsic infectiousness of the variant is $\rho$ times higher than that of the wild type and that both strains have perfect cross immunity against each other.
}

\DIFaddend Ignoring birth and death, the incidence of infection caused by the variant $\iv$ and the wild type $\iw$ \DIFdelbegin \DIFdel{assuming perfect cross immunity }\DIFdelend \DIFaddbegin \DIFadd{under constant-strength intervention }\DIFaddend is given by:
\begin{align}
\frac{dS}{dt} &= - \iv(t) - \iw(t)\\
\iv(t) &= \DIFdelbegin %DIFDELCMD < \Rv%%%
\DIFdelend \DIFaddbegin \DIFadd{\rho  }\hRw\DIFaddend (t) \DIFaddbegin \DIFadd{\frac{S(t)}{N} }\DIFaddend \int_{0}^\infty \iv(t-\tau) g_{\mathrm{var}}(\tau) \dtau\\
\iw(t) &= \DIFdelbegin %DIFDELCMD < \Rw%%%
\DIFdelend \DIFaddbegin \hRw\DIFaddend (t) \DIFaddbegin \DIFadd{\frac{S(t)}{N}  }\DIFaddend \int_{0}^\infty \iw(t-\tau) g_{\mathrm{wt}}(\tau) \dtau
\end{align}
where $S$ represents the \DIFdelbegin \DIFdel{proportion }\DIFdelend \DIFaddbegin \DIFadd{number }\DIFaddend of susceptible individuals \DIFdelbegin \DIFdel{.
We discretize the model at the time scale of 0.025 days as described in \mbox{%DIFAUXCMD
\cite{park2021forward} }\hspace{0pt}%DIFAUXCMD
and simulated the model with $\iv(0) = 0.001$ }\DIFdelend and \DIFdelbegin \DIFdel{$\iw(0) = 0.1$}\DIFdelend \DIFaddbegin \DIFadd{$N$ represents the population size.
For this model, the instantaneous reproduction numbers are defined as:
}\begin{align}
\DIFadd{\Rv(t) }&\DIFadd{= \rho  \hRw(t)S(t)/N}\\
\DIFadd{\Rw(t) }&\DIFadd{=\hRw(t) S(t) /N
}\end{align}
\DIFadd{In this case, the ratio between the instantaneous reproduction numbers (i.e., the relative strength) remain constant over time:
}\begin{equation}
\DIFadd{\frac{\Rv(t)}{\Rw(t)} = \rho.
}\end{equation}

\DIFadd{Likewise, the incidence of infection caused by the variant $\iv$ and the wild type $\iw$ under constant-speed intervention is given by:
}\begin{align}
\DIFadd{\frac{dS}{dt} }&\DIFadd{= - \iv(t) - \iw(t)}\\
\DIFadd{\iv(t) }&\DIFadd{= \rho  \Rw(0) \frac{S(t)}{N} \int_{0}^\infty \left[\iv(t-\tau) g_{\mathrm{var}}(\tau) \exp \left(- \int_0^\tau h(t-\sigma) \dsigma \right)\right] \dtau}\\
\DIFadd{\iw(t) }&\DIFadd{= \Rw(0) \frac{S(t)}{N} \int_{0}^\infty \left[\iw(t-\tau) g_{\mathrm{wt}}(\tau) \exp \left(- \int_0^\tau h(t-\sigma) \dsigma \right)\right] \dtau
}\end{align}
\DIFadd{where $h$ represents the time-varying hazard of isolation.
Following the definition in Equation~(2) of the main text, the instantaneous reproduction numbers are defined as:
}\begin{align}
\DIFadd{\Rv(t) }&\DIFadd{= \rho \Rw(0)  \frac{S(t)}{N} \int_{0}^\infty g_{\mathrm{var}}(\tau) \exp \left(- \int_0^\tau h(t-\sigma) \dsigma \right) \dtau }\\
\DIFadd{\Rw(t) }&\DIFadd{= \Rw(0)  \frac{S(t)}{N} \int_{0}^\infty g_{\mathrm{wt}}(\tau) \exp \left(- \int_0^\tau h(t-\sigma) \dsigma \right) \dtau
}\end{align}
\DIFadd{In this case, the ratio between the instantaneous reproduction numbers (i.e., the relative strength) will not remain constant unless both the wild type and the variant have identical generation-interval distributions.
}

\DIFadd{In Section 5 (Figure 3 in the main text), we rely on the constant-strength model for simplicity}\DIFaddend .
To allow for smooth \DIFdelbegin \DIFdel{changes }\DIFdelend \DIFaddbegin \DIFadd{decrease and increase }\DIFaddend in $\Rw(t)$ \DIFaddbegin \DIFadd{(and $\Rv(t)$) }\DIFaddend around day 30 and 60, we let:
\begin{equation}
\Rw(t) = 2 \times \left[1 - 0.8 \left(\frac{1}{2} + \frac{\arctan(t-30)}{\pi} \right) \right] \times \left[1 + 1.5 \left(\frac{1}{2} + \frac{\arctan(t-60)}{\pi} \right) \right].
\end{equation}
In this case, $\Rw(t)$ remains around 2 and slowly decreases to $0.4 = 2 \times (1-0.8)$ around day 30, and slowly increases back up to $1 = 0.4 \times (1 + 1.5)$ around day 60.
\DIFdelbegin \DIFdel{Finally, we assume $\Rv(t) = \rho \Rw(t)$.From simulated incidence curves between day 15 and 70, we estimate the }\DIFdelend \DIFaddbegin \DIFadd{We also present the analogous simulation with sharp changes in $\RR$ in Figure S1 using:
}\begin{equation}
\DIFadd{\Rw(t) = \begin{cases}
2 & t < 30\\
0.4 & 30 \leq t < 60\\
1 & 60 \leq t 
\end{cases}
}\end{equation}

\DIFadd{In Section 6 (Figure 5 in the main text), we compare the impact of constant-strength and constant-speed models in response to an emerging variant.
For simplicity, we only consider the introduction of interventions and not the lifting of interventions.
In both scenarios, we assume that the infections caused by the wild type is stable at the introduction of interventions (i.e., $\RR=1$ and $r=0$).
In the constant-strength scenario, we have: 
}\begin{equation}
\DIFadd{\Rw(t) = \begin{cases}
1 & t < 30\\
\theta & 30 \leq t
\end{cases}.
}\end{equation}
\DIFadd{In the constant-speed scenario, we have: 
}\begin{align}
\DIFadd{\Rw(0) }&\DIFadd{= 1}\\
\DIFadd{h(t) }&\DIFadd{= \begin{cases}
0 & t < 30\\
\phi & 30 \leq t
\end{cases}.
}\end{align}
\DIFadd{Both $\theta$ and $\phi$ are chosen such that the post-intervention }\DIFaddend instantaneous reproduction number \DIFdelbegin \DIFdel{using \eref{rt}.
We ignore incidence before day 15 to remove any potential transient effects.
}\DIFdelend \DIFaddbegin \DIFadd{of the variant is equal to 0.9 when its mean generation interval is 5 days, as in Figure 4 in the main text.
In both cases, we assume $\rho = 1.61$, which is the value we obtain for $\delta=0.1\pday$ \mbox{%DIFAUXCMD
\citep{davies2021estimated}}\hspace{0pt}%DIFAUXCMD
, $\rw=0\pday$, $\Gw = \Gv = 5\,\textrm{days}$, and $\kappa = 1/5$ \mbox{%DIFAUXCMD
\citep{ferretti2020quantifying}}\hspace{0pt}%DIFAUXCMD
.
}

\DIFadd{We discretize the model at the time scale of $\Delta t =0.025\ \textrm{days}$ as described in \mbox{%DIFAUXCMD
\cite{park2021forward}}\hspace{0pt}%DIFAUXCMD
.
The initial conditions are set such that infections caused by the variant and the wild type can exhibit exponential changes from $\iv(0) = 0.001$ and $\iw(0) = 0.1$ since day 0.
The population size is set to $N=40,000$
}\DIFaddend 

\pagebreak

\section*{Supplementary Figure}

\setcounter{figure}{0}    
\renewcommand\thefigure{S\arabic{figure}}    

\begin{figure}[!pht]
\begin{center}
\includegraphics[width=0.9\textwidth]{Rtbias.pdf}
\caption{
\textbf{Estimates of relative strength over time under different scenarios assuming step changes in reproduction number.}
See Figure 3 of the original text for figure caption.
}
\end{center}
\end{figure}


\pagebreak

\bibliography{newvariant_abbv.bib}

\end{document}
