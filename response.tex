\documentclass[12pt]{article}
\usepackage[utf8]{inputenc}

\usepackage{color}

\usepackage{xspace}

\usepackage{lmodern}

\usepackage{amsthm}
\usepackage{amsmath}
\usepackage{amssymb}
\usepackage{amsfonts}

\usepackage[pdfencoding=auto, psdextra]{hyperref}

\usepackage{natbib}
\bibliographystyle{chicago}

\newcommand{\eref}[1]{Eq.~(\ref{eq:#1})}
\newcommand{\fref}[1]{Fig.~\ref{fig:#1}}

%% Consistent, changeable style for subscripts
\newcommand{\vvvar}{\mathrm{var}}
\newcommand{\wwwt}{\mathrm{wt}}

\newcommand{\rx}[1]{\ensuremath{{r}_{#1}}\xspace} 
\newcommand{\ry}[1]{\rx{\mathrm{#1}}} 
\newcommand{\rw}{\rx{\wwwt}}
\newcommand{\rv}{\rx{\vvvar}}

\newcommand{\Rx}[1]{\ensuremath{{\mathcal R}_{#1}}\xspace} 
\newcommand{\Ry}[1]{\Rx{\mathrm{#1}}}
\newcommand{\Ro}{\Rx{0}}
\newcommand{\RR}{\ensuremath{{\mathcal R}}\xspace}
\newcommand{\Rw}{\Rx{\wwwt}}
\newcommand{\Rv}{\Rx{\vvvar}}

\newcommand{\dd}[1]{\ensuremath{\, \mathrm{d}#1}}
\newcommand{\dtau}{\dd{\tau}}
\newcommand{\dx}{\dd{x}}
\newcommand{\dsigma}{\dd{\sigma}}

\newcommand{\rev}{\subsection*}
\newcommand{\revtext}{\textsf}
\setlength{\parskip}{\baselineskip}
\setlength{\parindent}{0em}

\newcommand{\comment}[3]{\textcolor{#1}{\textbf{[#2: }\textsl{#3}\textbf{]}}}
\newcommand{\jd}[1]{\comment{cyan}{JD}{#1}}
\newcommand{\swp}[1]{\comment{magenta}{SWP}{#1}}
\newcommand{\dc}[1]{\comment{blue}{DC}{#1}}
\newcommand{\jsw}[1]{\comment{green}{JSW}{#1}}
\newcommand{\hotcomment}[1]{\comment{red}{HOT}{#1}}

\newcommand{\psymp}{\ensuremath{p}} %% primary symptom time
\newcommand{\ssymp}{\ensuremath{s}} %% secondary symptom time
\newcommand{\pinf}{\ensuremath{\alpha_1}} %% primary infection time
\newcommand{\sinf}{\ensuremath{\alpha_2}} %% secondary infection time

\newcommand{\psize}{{\mathcal P}} %% primary cohort size
\newcommand{\ssize}{{\mathcal S}} %% secondary cohort size

\newcommand{\gtime}{\tau_{\rm g}} %% generation interval
\newcommand{\gdist}{g} %% generation-interval distribution
\newcommand{\idist}{\ell} %% incubation period distribution

\newcommand{\total}{{\mathcal T}} %% total number of serial intervals


\begin{document}

\noindent Dear Editor:

Thank you for the chance to revise and resubmit our manuscript. 
We have made revisions to our manuscript to address the reviewers' comments.
Overall, we have tried to provide a more thorough review of the literature and highlight the novelty of our work while acknowledging its limitations.
We have also added a new figure as suggested by Reviewer 3.
Below please find our detailed responses to reviewers.

\rev{Reviewer \#1}

\revtext{This is a review of ``Roles of generation-interval distributions in shaping relative epidemic strength, speed, and control of new SARS-CoV-2 variants'' by Park and others for Royal Society Interface.  This is the latest in an ongoing series of papers (Champredon 2015, Champredon 2018, Park 2019, Dushoff 2021, ...) in which the last author has considered aspects of incubation, growth, and reproduction in SIR epidemic theory.  In this manuscript, the authors use theory to anticipate the effects of COVID-19 variants where the distribution of infectivity evolves.}

\revtext{The manuscript is excellently written. The figures are clear.  All the mathematics seems satisfactory.  Unfortunately, the manuscript does not make its case for novel results.}

Thank you for the review.
We have tried to frame our results better, and to clarify both the novelty and importance of this study.
The primary purpose of our study is two-fold: (1) to illustrate the impact of uncertainty in the generation interval in shaping the population-level transmission advantage of new variants and (2) to evaluate when it is appropriate to hold relative strength or speed constant.
While several studies of new variants have assumed constant relative speed (e.g., by fitting a logistic growth model), we argue that holding relative strength fixed is generally more appropriate given that the dynamics of SARS-CoV-2 have been driven primarily by strength-like interventions.
We also show that this relationship changes under speed-like interventions---that is, relative strength changes but relative speed remains constant.
We believe that the addition of a new simulations (Figure 5) and the revised text better highlight the novelty of our work.

\revtext{The title is overly-broad as it suggests consideration of general forms of generation interval distributions (along the lines of Miller 2010 10.1016/j.jtbi.2009.08.007), when infect the authors study variation around the time of peak infectivity using a (very standard) $\gamma$-chain trick.  The study of generation time and $\gamma$ chains has a long history from the cited-but-unreviewed Svensson 2007, Wallinga 2007, to uncited Kenah 2008, A.  Lloyd 2001, D. Anderson 1980, Bailey 1964 (see Park 2019 and Champredon 2015).  In light of these and other works, the title and abstract fail to identify novel contributions to our understanding of epidemic theory.}

We have changed the title and the abstract to highlight the novelty of our work. 
See also below.

\revtext{Although the authors are aware of the history of research on this topic, they do little to review the relevant material, leaving this manuscript feeling like a superficial re-hashing of obvious material.  The introduction doesn't provide any serious review of the theory}

We now provide a review of this topic in the introduction along with a more detailed justification of the assumptions of our work:

``Epidemic speed can be typically estimated from incidence of infection during the exponential growth period \citep{mills2004transmissibility,nishiura2009transmission,ma2014estimating}, but epidemic strength is difficult to measure directly.
Instead, epidemic strength is often inferred from the observed epidemic speed using the generation-interval distribution, an approach popularized by \citep{wallinga2007generation}.
The generation interval is defined as the time between infection and transmission and provides information about the time scale of individual-level transmission \citep{svensson2007note}.
The generation interval is also distinctly different from other so-called ``transmission intervals'' that measure time between successive infections---this includes the serial interval, which is defined as the time between symptom onsets in an infector-infectee pair \citep{fine2003interval,grassly2008mathematical,britton2019estimation,ali2020serial,park2021forward}.

The exact shape of the distribution depends on several factors---including the shape of latent and infectious periods \citep{lloyd2001realistic,wearing2005appropriate,roberts2007model} as well as more detailed life history of a disease \citep{huber2016quantitative}---and thus can be difficult to estimate.
While it is possible to consider general forms of generation-interval distributions \citep{miller2010epidemics,svensson2015influence}, summarizing the distribution in terms of its mean and variability---for example, by using the Gamma distribution---can still provide a robust link between epidemic speed and strength for real diseases and yield important biological insights \citep{park2019practical}.
In particular, several studies have noted, in many contexts, that mechanisms that increase the mean generation interval increase epidemic strength $\RR$ given observed epidemic speed $r$ \citep{eaton2014proportion,powers2014impact,weitz2015modeling,park2020time}.''

Assuming Gamma-distributed generation intervals is different from the standard Gamma-chain trick. We now explain this later in Section 2:

``Various Gamma-generation-interval assumptions have been widely used in epidemic modeling, including for models of SARS-CoV-2 \citep{doi:10.1098/rsif.2020.0144}.
The Gamma-generation-interval assumption includes as a special case models that assume exponentially distributed generation intervals (when $\kappa=1$), corresponding to the SIR model \citep{anderson1991infectious}.
We note that when the infectious periods are Gamma distributed---another standard assumption in epidemic modeling---the resulting generation interval does not follow the Gamma distribution;
in particular, the mean generation interval can be shorter than the mean infectious period because transmission occurs throughout the infectious period until recovery (see \cite{roberts2007model} for detailed discussion).''

\revtext{Section 4 doesn't give references to prior work, and section 5 ignores almost all of the literature on the topic of estimating R from incidence data.}

We have added references in Section 4.
We have also added a new paragraph in the beginning of Section 5 reviewing previous work on estimating R from incidence data:

``Instead of inferring relative strength from speed, one can directly estimate time-varying reproduction numbers $\RR(t)$ of the variant and the wild type from incidence data and directly compare their ratios when the incidence of infection and the generation-interval distributions are known---such methods have been used in previous analyses of the alpha variant by \cite{volz2021transmission}.
Broadly, there are two types of time-varying reproduction numbers: case reproduction number and instantaneous reproduction number.
The case reproduction number is defined as the average number of secondary infections caused by an individual infected at time $t$ and therefore depends on transmission after time $t$ \citep{wallinga2004different}.
The instantaneous reproduction number is defined as the average number of secondary infections that would be caused by an individual infected at time $t$ if conditions were to remain the same \citep{fraser2007estimating}; 
therefore, the instantaneous reproduction number only depends on transmission at time $t$ and is most appropriate for real-time evaluation of changes in transmission \citep{gostic2020practical}.
The estimation of the instantaneous reproduction number was popularized by \cite{cori2013new} and has been widely adopted in epidemiological analyses of SARS-CoV-2 \citep{abbott2020estimating,knight2020estimating,flaxman2020Rt,brauner2021inferring,li2021temporal}.''

\revtext{Other comments:}

\revtext{Topic on which there is a considerable confusion in terminology in the literature.  In addition to "generation interval", terms "generation time", "transmission interval", "serial interval", and others have been used in the literature, all with related by different definitions.  This manuscripts definition "time between infection and transmission" implies a single transmission event, but Kenah pointed out that time-contraction appears when allowing multiple transmission events (see Champredon 2015).  A clearer definition up front will help readers.  Fortunately, the results don't depend on the specific definition.}

Thank you for the suggestions. We have tried to clarify what we mean by the generation interval. In particular, often-mentioned time contractions of generation intervals apply to the forward realized generation interval, which measure time between actual infection events, rather than the intrinsic generation interval, which measure the intrinsic infectiousness of infected individuals. Under strength-like interventions, the intrinsic generation interval does not vary across time, even though the realized generation interval contracts. These distinctions are explained in Champredon 2015, and we have tried to clarify these ideas again here in Section 2:

``Under constant-strength changes that reduce the transmission rate by a constant amount regardless of the age of infection, the instantaneous generation-interval distribution does not change across time \citep{fraser2007estimating}.
In this case, the instantaneous generation-interval distribution is also often referred to as the intrinsic generation-interval distribution \citep{champredon2015intrinsic,champredon2018two,gostic2020practical,park2020time}---
for example, the standard SEIR model can be equivalently expressed as a renewal equation with time-invariant intrinsic generation-interval distribution $g(\tau)$ as shown in \citep{champredon2018equivalence}.
The instantaneous generation-interval distribution is also different from the realized generation-interval distribution, which measures time between actual infection events (see \citep{champredon2015intrinsic}).
Previous studies have noted, in many contexts, that the realized generation intervals can contract due to susceptible depletion---a special case of constant-strength changes \citep{kenah2008generation,nishiura2010time,champredon2015intrinsic}---even though the instantaneous generation-interval distribution remains unchanged in this scenario.
While the instantaneous generation-interval distribution can change under speed-like changes (see \cite{fraser2007estimating} and Section 6--7 for detailed discussions), assuming a time-invariant instantaneous generation-interval distribution is often appropriate in the context of SARS-CoV-2, given that the dynamics of its spread have been primarily driven by strength-like intervention, such as social distancing measures \citep{flaxman2020Rt} and vaccination \citep{moore2021vaccination}.
Indeed, many dynamical models of SARS-CoV-2 infections have solely relied on constant-strength changes, either implicitly or explicitly assuming a time-invariant intrinsic generation-interval distribution (e.g., \citep{flaxman2020Rt,gostic2020practical,brauner2021inferring}).
Therefore, we neglect changes in the intrinsic generation-interval distribution over time for now and focus on the impact of constant-strength changes on the inference of dynamics of new SARS-CoV-2 variants.
We revisit these ideas in Section 6 and compare the effects of constant-speed interventions with those of constant-strength interventions.''

\revtext{The 'speed'-'strength' distinction from the previous paper is seductive, but perhaps leaning into general quantitative illiteracy.  When one pays attention to units -- "speed" L/T,  "Growth rate" 1/T while "strength" has dimensions of force, typically, but reproduction number is a dimensionless ratio of cases per case.  Beyond time-independent final-size relations (which are themselves not robust to effects like behavior change), several researchers have argued that growth rates generally more useful than reproduction numbers (giving equivalent threshold conditions, for example) and not as prone to mis-interpretation.  Other authors in evolutionary biology and economics have argued that the reproduction number's scientific value is best salvaged using the "discounted reproduction success" (McNamara, Houston, and Collins 2001), rather than fighting of an R-r straw-man.}

We are certainly not trying to lean in to innumeracy. We agree that ``strength'' and ``speed'' are not used in the classical physics sense---instead, we try to be clear how we define them (and have tried to clarify further).

We (the first and the last author) have argued (\url{https://doi.org/10.1098/rspb.2020.1556}) that large segments of the epidemiology and modeling community have an over-emphasis on R, and have tried to use the terms speed and strength to underline our contention that these are better seen as complementary perspectives (and to link them to complementary perspectives on control).
As we argue here and elsewhere, we believe that sometimes one perspective over another gives a clearer picture.
For example, in the beginning of an epidemic, it is more appropriate to fix epidemic speed $r$ when evaluating uncertainties in the generation-interval distribution.
On the other hand, it is more often appropriate to fix relative strength $\rho$, rather than relative speed $\delta$, when conditions are changing due to interventions.
We hope that our revisions, including the addition of a new figure (see response to reviewer 3), have made these points clearer.
We thus decided to keep the terms ``speed'' and ``strength'' in this manuscript.

\revtext{Equation 1 As Kermack and McKendrick introduced it, the compartmental model is a convenient *specialization* of the integral models-- see the nice re-analysis by Breda et al., 2012.}

We now cite Breda et al., and acknowledge that ``This framework provides a flexible way of modeling disease dynamics and generalizes a wide range of compartmental models, including the SEIR model \citep{heesterbeek1996concept, diekmann2000mathematical, roberts2004modelling, aldis2005integral,breda2012formulation, champredon2018equivalence}.''

\revtext{Equation 4 assumes the wild-type and variant co-dominate the population.  This may not be the cases when multiple variants are co-circulating, particularly during replacement events.}

We now acknowledge this point and provide a citation for a study that discusses this point:

``When more than two strains are co-circulating, the picture is more complicated \citep{campbell2021increased}; we focus here on comparing two strains at a time.''

\revtext{p 4 l55.  Shape of the distribution matters -- heavy tails and failed central concentration.  This detail will escape the naive reader unless we poke them a little more with it.}

We have reworded this section:

``Here, we focus on differences in mean generation intervals, assuming that both the squared coefficient of variation and distribution shape of the variant are otherwise the same as for the wild type: $\kappa_{\mathrm{wt}} = \kappa_{\mathrm{var}} = \kappa$.
Changes in shape can be important \citep{miller2010epidemics,svensson2015influence}, and we do not investigate them here. 
We do note that, loosely, we expect a distribution with higher coefficient of variation to allow for more early transmission, and thus to have qualitatively similar effects to a distribution with a shorter mean in many cases \citep{park2019practical}.''

\rev{Reviewer \#2}

\revtext{This is a well-written paper about the role of the generation interval distribution in the estimation of the relative strength (reproduction number) and speed
(epidemic growth rate) of a new variant of a disease. Using a renewal equation
framework, they consider inferring relative strength from relative speed and vice
versa as well as inferring relative strength from incidence data. While all of the
analysis in the paper is sound, I am concerned that the strong assumptions made
(mass-action epidemic, constant generation interval distribution, etc.) will limit
the reliability of the results in a practical application. However, they clearly
establish that assumptions about the generation interval distribution have important implications for the analysis of novel variants. While I am generally
skeptical of population-level analyses of infectious disease transmission based
on generation intervals, these are done frequently so this is a useful insight.
More detailed comments are below.}

Thank you for the review. 
We have tried to make the limitation and assumptions clearer throughout the text.

\revtext{1. (page 3, first paragraph) It is important to acknowledge that the renewal equation approach to the relationship between the generation interval
distribution, R, and the growth rate r assumes a mass-action epidemic.
When we assume a fixed generation interval distribution, we must add the
assumption that the rate of depletion of susceptibles over time is negligible and the prevalence of infection is low. One of the most important
findings of Ref 10 is that the mean generation interval contracts when the
rate of depletion of susceptibles increases. While the renewal equation
does allow for the generation interval to change over time, the analyses
in Sections 3–5 assume fixed mean generation intervals. Because variants
are likely to spread when susceptibles are being depleted by both variants
(due to cross-protection) and when the prevalence of infection is not low, these implicit assumptions place significant limitations on the practical
implications of the analysis.}

We acknowledge that we are assuming homogeneous mixing---we feel that this is a clearer explanation because mass-action is often used to refer to how the transmission rate, beta, scales with the population size.

We note that we primarily focus on short-term dynamics until Section 4.
In this case, we can approximate the dynamics using exponential growth/decay, and therefore the generation-interval distribution is expected to remain fixed:

``Over a short period of time, we can assume that epidemiological conditions remain roughly constant: $\RR_x(t) \approx \RR_x$ and $g_x(t, \tau) \approx g_x(\tau)$, in which case the incidence of infections changes exponentially.
Here, we use the term ``epidemiological conditions'' to broadly refer to all factors that affect transmission---mathematically, they are captured by the kernel $K(t, \tau)$.
In the context of SARS-CoV-2 infections, we are essentially assuming that the changes in susceptible pool, behavior, and contact rates are usually small over a short period of time.
Then, the incidence of each strain grows (or decays) exponentially at rate $r_x$, satisfying the Euler-Lotka equation \citep{wallinga2007generation}:
\begin{equation}
\frac{1}{\RR_x} = \int_0^\infty \exp(- r_x \tau) g_x(\tau) \dtau.
\end{equation}''

We further note that a fixed generation-interval distribution does not rely on the assumption that the rate of depletion of susceptibles over time is negligible.
This is due to an ongoing confusion between the intrinsic and realized generation interval as we explained above.
For example, as shown in Champredon et al 2015 and 2018, the standard SEIR model can be equivalently written in terms of the renewal equation using a fixed \emph{intrinsic} generation-interval distribution, and we can still observe the contraction of the \emph{realized} generation interval for this model.
We have added a paragraph in Section 2 to clarify this distinction (please see response to reviewer 1 for text).

\revtext{(page 2, lines 55–60) It would help to clarify what is meant by “strengthlike” and “speed-like” in interventions. From the examples, it appears that
speed-like interventions work by identifying potential infections quickly
(effectively ending the infectious period early while not necessarily altering
the hazard of infectious contact) and strength-like interventions work by
reducing the hazard of infectious contact (while not necessarily altering
the duration of infectiousness). A good explanation of these ideas is on
lines 228–231 on page 10.}

We now define them more clearly throughout the text, including in the introduction:

``For example, when epidemic strength $\RR$ is fixed, assuming longer generation intervals leads to a slower epidemic growth (lower $r$), making the epidemic look easier to control with a constant-speed intervention, which isolates infected individuals at a constant hazard and therefore affects late transmission disproportionately.
As shown in \cite{doi:10.1098/rspb.2020.1556}, the hazard of isolation must be greater than the epidemic speed $r$ to reach the $r=0$ threshold.
In this same case, however, the uncertainty in the generation-interval distribution is expected to have no effects on the effectiveness of a constant-strength intervention, which reduces the hazard of infectious contacts and therefore affects transmission by an equal amount regardless of age of infection.
Conversely, when epidemic speed $r$ is fixed, assuming longer generation intervals leads to a higher estimate of epidemic strength ($\RR$), making the epidemic look harder to control with a constant-strength interventions.
Constant strength and speed are idealized representations of real-life interventions, which can be either strength-like (e.g., vaccination and social distancing) or speed-like (e.g., contact tracing and isolation) depending on how their effectiveness varies across the generation interval (see \cite{doi:10.1098/rspb.2020.1556} and Discussion).''

\revtext{(page 5, lines 126–128 and 137) It would help to point out that you assume
rwt < rvar when you introduce the five scenarios.}

Done.

\revtext{(page 9, Figure 3) I would order the rows in order of the assumed mean
generation interval of the variant: 4 days at the top, 5 days in the middle,
and 6 days on the bottom.}

Done.

\revtext{(page 13, lines 282–285) While contact tracing and household studies are
susceptible to a number of biases, they are likely to be a far more reliable
source of information about the transmissibility and generation intervals
of variants than population-level analyses of incidence data.}

We have now re-written this part:

``Although these estimates are more consistent with the attack rate analysis \citep{ukinvest},
we do not claim that they are necessarily more accurate---
while individual-level data from contact tracing can provide a more reliable source of information about the transmissibility and time scale of transmission in some cases, they can also be biased towards particular types of contacts---for example, household contacts are probably more likely to be identified---which could also affect the estimate of $\rho$.''

\revtext{(page 15, References) Curly brackets might need to be added to some of
the BibTeX entries to enforce correct capitalization (e.g., Ref. 9).}

Done.

\rev{Reviewer \#3}

\revtext{In this manuscript, Park and colleagues present a thoughtful analysis of how assumptions about the generation interval distribution for wild-type and variant SARS-CoV-2 can bias estimates of the relative growth rate ("speed") and relative reproduction number ("strength") of the variant. The analysis is mathematically sound and offers a thorough investigation of an important concept that has been the subject of much discussion in the epidemiological community but, to my knowledge, hasn't been given the formal treatment it deserves until now. My comments are mostly minor and intended to make the findings more accessible and transferable to epidemiological practice.}

Thank you for the review.

\revtext{Abstract: 37: On first reading, the distinction between 'speed-like' and 'strength-like' interventions wasn't clear to me, nor why contact tracing should target r but not R, and why social distancing should target R but not r. Could you use more recognizable terms here to highlight how these interventions differ from one another? (is the key difference that contact tracing reduces the number of infectious individuals and thus the probability that a susceptible person comes into contact with an infectious person, while social distancing reduces the per-contact chance of transmission? Or something else?) This comes up again in line 57 of the Introduction.}

We have tried to make these distinctions clearer throughout text, including in the introduction (see response to reviewer 2). We now begin by introducing constant-strength and constant-speed interventions, which have clear definitions, and generalize them (conceptually) to explain speed- and strength-like interventions.

\revtext{166: I think it would be worth stating the range of Rwt and/or Rvar that yields this range for delta (I think it's something like 0.49-1.27 for the wild type, 0.78-2.04 for the variant, based on Figure 2 - but would be helpful if the reader didn't have to derive). Is the varying R what you're referring to when you speak of "underlying epidemiological conditions" in 166-167?}

We now clarify this when it is mentioned for the first time:

``Over a short period of time, we can assume that epidemiological conditions remain roughly constant: $\RR_x(t) \approx \RR_x$ and $g_x(t, \tau) \approx g_x(\tau)$, in which case the incidence of infections changes exponentially.
Here, we use the term ``epidemiological conditions'' to broadly refer to all factors that affect transmission---mathematically, they are captured by the kernel $K(t, \tau)$.
In the context of SARS-CoV-2 infections, we are essentially assuming that the changes in susceptible pool, behavior, and contact rates are usually small over a short period of time.''

\revtext{184: I think that "inference data" should be "incidence data"}

Done.

\revtext{Figure 4: I found this figure quite difficult to parse and not especially informative. Are the vertical bars the means of these distributions? If so, they should be labeled and referenced as such in the caption. Also in the figure 4 caption: the term ``generation intervals'' (plural) struck me as a bit strange - I think the singular would make more sense and be more consistent with what's used in the literature. Also, it would be helpful to put theta and phi in the vertical axis labels for E and F (e.g., "Strength (theta)" and "Speed (phi; 1/days)"). Unless this is observed speed and strength, meaning Rvar and rvar? In E, I'm also confused why the strength of the "constant-strength" line is increasing - shouldn't it be flat? Or what have I missed? I think some more details to refer back to the variables in the text might be helpful - for example, I think that (F) is the "Pre-intervention (black) and post-intervention speed of variant transmission (rvar) conditional on the mean generation interval of the variant (Gvar, colored)."}

We have tried to make the figure caption clearer.

\revtext{Really what I would like to see here (Figure 4) is some simulated epidemic curves and perhaps final epidemic sizes - I think this would go much further to communicate the intuition behind how the speed-based and strength-based interventions differ than what's currently presented. One of the main conclusions of the paper seems to be that speed-based interventions could be more effective than strength-based interventions when the mean generation interval is long; presenting such simulations would be a good way to drive this point home and to give intuition for just how different speed-based vs strength-based interventions actually are.}

We have added Figure 5 and some additional text to explain the differences between strength- and speed-based interventions.

\revtext{264: "epidemiological dynamics" is used again here - what specifically do you mean?}

We have rewritten this part:

``these biases may be assessed by considering whether estimates of relative strength appear to vary systematically with the direction of changes in the incidence of infections caused by the variant.''

\revtext{265: "can also lead to different conclusions about the effectiveness of interventions" - what sort of different conclusions? In which circumstances?}

We have added further explanations:

``Finally, differences in generation intervals can also lead to different conclusions about the effectiveness of interventions.
If the variant has longer generation intervals than the wild type, speed-like interventions will be relatively more effective than naive estimates would suggest. 
Conversely, strength-like intervention will be relatively more effective if the variant has shorter generation intervals.''

\revtext{271: "If new variants has a longer infectious period" (typo)}

Done.

\revtext{Missing equation reference in the Supplement.}

Done.

\bibliography{newvariant_abbv.bib}

\end{document}
