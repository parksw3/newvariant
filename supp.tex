\documentclass[12pt]{article}
\usepackage[top=1in,left=1in, right = 1in, footskip=1in]{geometry}

\usepackage{graphicx}
\usepackage{xspace}
%\usepackage{adjustbox}

\newcommand{\comment}{\showcomment}
%% \newcommand{\comment}{\nocomment}

\newcommand{\showcomment}[3]{\textcolor{#1}{\textbf{[#2: }\textsl{#3}\textbf{]}}}
\newcommand{\nocomment}[3]{}

\newcommand{\jd}[1]{\comment{cyan}{JD}{#1}}
\newcommand{\swp}[1]{\comment{magenta}{SWP}{#1}}
\newcommand{\bmb}[1]{\comment{blue}{BMB}{#1}}
\newcommand{\djde}[1]{\comment{red}{DJDE}{#1}}

\newcommand{\eref}[1]{Eq.~\ref{eq:#1}}
\newcommand{\fref}[1]{Fig.~\ref{fig:#1}}
\newcommand{\Fref}[1]{Fig.~\ref{fig:#1}}
\newcommand{\sref}[1]{Sec.~\ref{#1}}
\newcommand{\frange}[2]{Fig.~\ref{fig:#1}--\ref{fig:#2}}
\newcommand{\tref}[1]{Table~\ref{tab:#1}}
\newcommand{\tlab}[1]{\label{tab:#1}}
\newcommand{\seminar}{SE\mbox{$^m$}I\mbox{$^n$}R}

\usepackage{amsthm}
\usepackage{amsmath}
\usepackage{amssymb}
\usepackage{amsfonts}

\usepackage{lineno}
\linenumbers

\usepackage[pdfencoding=auto, psdextra]{hyperref}

\usepackage{natbib}
\setcitestyle{numbers} 
\setcitestyle{square}
\bibliographystyle{prsb}
\date{\today}

\usepackage{xspace}
\newcommand*{\ie}{i.e.\@\xspace}

\usepackage{color}

%% Consistent, changeable style for subscripts
\newcommand{\vvvar}{\mathrm{var}}
\newcommand{\wwwt}{\mathrm{wt}}

\newcommand{\rx}[1]{\ensuremath{{r}_{#1}}\xspace} 
\newcommand{\ry}[1]{\rx{\mathrm{#1}}} 
\newcommand{\rw}{\rx{\wwwt}}
\newcommand{\rv}{\rx{\vvvar}}

\newcommand{\Rx}[1]{\ensuremath{{\mathcal R}_{#1}}\xspace} 
\newcommand{\Ry}[1]{\Rx{\mathrm{#1}}}
\newcommand{\Ro}{\Rx{0}}
\newcommand{\RR}{\ensuremath{{\mathcal R}}\xspace}
\newcommand{\Rw}{\Rx{\wwwt}}
\newcommand{\Rv}{\Rx{\vvvar}}

\newcommand{\days}{\ensuremath{\, \textrm{days}}}
\newcommand{\pday}{\ensuremath{/\textrm{day}}}
\newcommand{\dd}[1]{\ensuremath{\, \mathrm{d}#1}}
\newcommand{\dtau}{\dd{\tau}}
\newcommand{\dx}{\dd{x}}
\newcommand{\dsigma}{\dd{\sigma}}

\newcommand{\ix}[1]{\ensuremath{{i}_{#1}}\xspace} 
\newcommand{\iy}[1]{\ix{\mathrm{#1}}}
\newcommand{\iw}{\ix{\wwwt}}
\newcommand{\iv}{\ix{\vvvar}}

\newcommand{\Gx}[1]{\ensuremath{{\bar G}_{#1}}\xspace} 
\newcommand{\Gy}[1]{\Gx{\mathrm{#1}}}
\newcommand{\Gw}{\Gx{\wwwt}}
\newcommand{\Gv}{\Gx{\vvvar}}

\newcommand{\tsub}[2]{#1_{{\textrm{\tiny #2}}}}
\newcommand{\tstart}{\ensuremath{\tsub{t}{start}}\xspace}
\newcommand{\tend}{\ensuremath{\tsub{t}{end}}\xspace}

\newcommand{\betaeff}{\ensuremath{\tsub{\beta}{eff}}\xspace}
\newcommand{\Keff}{\ensuremath{\tsub{K}{eff}}\xspace}

\newcommand{\pt}{p} %% primary time
\newcommand{\st}{s} %% secondary time

\newcommand{\psize}{{\mathcal P}} %% primary cohort size
\newcommand{\ssize}{{\mathcal S}} %% secondary cohort size

\newcommand{\gtime}{\sigma} %% generation interval
\newcommand{\gdist}{g} %% generation-interval distribution

\newcommand{\geff}{g_{\textrm{eff}}} %% generation-interval distribution

\newcommand{\total}{{\mathcal T}} %% total number of serial intervals

\newcommand{\PP}{{\mathcal P}}
\newcommand{\II}{{\mathcal I}}

\begin{document}

\begin{flushleft}{
	\Large
	\textbf\newline{
		Roles of generation-interval distributions in shaping relative epidemic strength, speed, and control of new SARS-CoV-2 variants---Supplementary Materials
	}
}
\newline
\\
Sang Woo Park\textsuperscript{1,*}
Benjamin M.\ Bolker\textsuperscript{2,3,4}
Sebastian Funk\textsuperscript{5,6}
C.\ Jessica E.\ Metcalf\textsuperscript{1,7}
Joshua S.\ Weitz\textsuperscript{8,9}
Bryan T.\ Grenfell\textsuperscript{1,7,10}
Jonathan Dushoff\textsuperscript{2,3,4}
\\
\bigskip
\textbf{1} Department of Ecology and Evolutionary Biology, Princeton University, Princeton, NJ, USA
\\
\textbf{2} Department of Biology, McMaster University, Hamilton, ON, Canada
\\
\textbf{3} Department of Mathematics and Statistics, McMaster University, Hamilton, ON, Canada
\\
\textbf{4} M.\,G.\,DeGroote Institute for Infectious Disease Research, McMaster University, Hamilton, ON, Canada
\\
\textbf{5} Department for Infectious Disease Epidemiology, London School of Hygiene and Tropical Medicine, London, UK
\\
\textbf{6} Centre for Mathematical Modelling of Infectious Diseases, London School of Hygiene and Tropical Medicine, London, UK
\\
\textbf{7} Princeton School of Public and International Affairs, Princeton University, Princeton, NJ, USA
\\
\textbf{8} School of Biological Sciences, Georgia Institute of Technology, Atlanta, GA, USA
\\
\textbf{9} School of Physics, Georgia Institute of Technology, Atlanta, GA, USA
\\
\bigskip

*Corresponding author: swp2@princeton.edu
\bigskip

\end{flushleft}

%% 6635 words

\section{Supplementary Text}

\subsection{Measuring relative growth for multiple co-circulating strains}

In order to model multiple co-circulating strain, we write $i_x(t)$ to represent the incidence of strain $x$ (for $x = 1, \dots, N$).
When incidence is changing exponentially ($i_x(t) = i_x(t_0) \exp(r_x t)$), the proportion of strain $y$ in the population can be written as:
\begin{equation}
p_y(t) = \frac{i_y(t_0) \exp(r_y t)}{\sum_{x=1}^N i_x(t_0) \exp(r_x t)}.
\end{equation}
Logit transforming $p_y(t)$ then gives:
\begin{align}
\mathrm{logit}(p_y(t)) &= \log\left(\frac{p_y(t)}{1-p_y(t)}\right)\\
&= \log\left(\frac{i_y(t_0) \exp(r_y t)}{\sum_{x \neq y} i_x(t_0) \exp(r_x t)}\right)\\
&= \log(i_y(t_0)) + r_y t - \log \left(\sum_{x \neq y} i_x(t_0) \exp(r_x t)\right)
\end{align}
which is a nonlinear function of time $t$.
Therefore, the growth rate measured at the logit scale is expected to change over time when multiple strains are co-circulating. 
Instead, a multinomial logistic regression model is required to infer relative speed \citep{campbell2021increased}.

\subsection{Two-strain renewal equation}

We use a two-strain renewal equation to simulate the spread of the variant and the wild type.
Ignoring birth and death, the incidence of infection caused by the variant $\iv$ and the wild type $\iw$ assuming perfect cross immunity is given by:
\begin{align}
\frac{dS}{dt} &= - \iv(t) - \iw(t)\\
\iv(t) &= \Rv(t) \int_{0}^\infty \iv(t-\tau) g_{\mathrm{var}}(\tau) \dtau\\
\iw(t) &= \Rw(t) \int_{0}^\infty \iw(t-\tau) g_{\mathrm{wt}}(\tau) \dtau
\end{align}
where $S$ represents the proportion of susceptible individuals.
We discretize the model at the time scale of 0.025 days as described in \cite{park2021forward} and simulated the model with $\iv(0) = 0.001$ and $\iw(0) = 0.1$.
To allow for smooth changes in $\Rw(t)$ around day 30 and 60, we let:
\begin{equation}
\Rw(t) = 2 \times \left[1 - 0.8 \left(\frac{1}{2} + \frac{\arctan(t-30)}{\pi} \right) \right] \times \left[1 + 1.5 \left(\frac{1}{2} + \frac{\arctan(t-60)}{\pi} \right) \right].
\end{equation}
In this case, $\Rw(t)$ remains around 2 and slowly decreases to $0.4 = 2 \times (1-0.8)$ around day 30, and slowly increases back up to $1 = 0.4 \times (1 + 1.5)$ around day 60.
Finally, we assume $\Rv(t) = \rho \Rw(t)$.
From simulated incidence curves between day 15 and 70, we estimate the instantaneous reproduction number using \eref{rt}.
We ignore incidence before day 15 to remove any potential transient effects.

\pagebreak

\section*{Supplementary Figure}

\setcounter{figure}{0}    
\renewcommand\thefigure{S\arabic{figure}}    

\begin{figure}[!pht]
\begin{center}
\includegraphics[width=0.9\textwidth]{Rtbias.pdf}
\caption{
\textbf{Estimates of relative strength over time under different scenarios assuming step changes in reproduction number.}
See Figure 3 of the original text for figure caption.
}
\end{center}
\end{figure}


\pagebreak

\bibliography{newvariant_abbv.bib}

\end{document}
